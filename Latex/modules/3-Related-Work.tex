\section{Verwandte Arbeiten}\label{RelatedWork}

Da das MST Problem eines der grundlegendsten Graphenprobleme ist, wird sich schon lange damit beschäftigt, effiziente Algorithmen für dieses Problem zu finden. Seitdem die bereits erwähnten Algorithmen von Kruskal \cite{kruskal1956shortest}, Jarnik und Prim \cite{prim1957shortest} sowie \boruvka \cite{boruuvka1926jistem} entwickelt wurden, gibt es viele Arbeiten, die darauf aufbauen.

So hat zum Beispiel in 1995 Karger et al. \cite{karger1995randomized} den KKT-Algorithmus entworfen. Dieser kombiniert \boruvka zusammen mit randomisierten Stichproben und effektivem Kanten filtern, sodass dieser eine erwartete lineare Laufzeit erreicht hat.
Ansonsten hat Osipov et al. \cite{osipov2009filter} in 2009 den Filter-Kruskal Algorithmus entwickelt. 
Dieser Algorithmus verbessert in der Praxis den hohen Aufwand des Kanten Sortierens bei Kruskal, weshalb wir ihn auch für unsere lokale Berechnungen weiter verwenden. 

Natürlich gibt es mittlerweile auch eine Vielzahl an parallelen MST Algorithmen. In 2021 hat Dhulipala \cite{dhulipala2021theoretically} eine shared-memory Varainte von \boruvkasAlgorithmus entwickelt. Wobei erst letztes Jahr Esfahani et al. \cite{koohi2022mastiff} einen \textit{structure aware} Algorithmus veröffentlicht hat, der diesen noch übertrifft.

Neben diesen shared-memory Varianten gibt es noch weitere parallele Algorithmen, die wie wir auf geteiltem Speicher arbeiten. So haben Chung et al. \cite{chung1996parallel} in 1996 erstmalige eine verteilte Versions von \boruvkasAlgorithmus entwickelt. Nur zwei Jahre später veröffentlichten Dehne und Götz \cite{dehne1998practical} ihre Algorithmen, auf denen wir mit unserer Arbeit aufbauen.
Im selben Jahr veröffentlichten auch Adler et al. \cite{adler1998communication} eine ähnliche Reihe an Kommunikationsoptimalen Algorithmen.

Im Gegensatz zu diesen Algorithmen, die für dichte Graphen konstruiert, wurden haben Sanders und Schimek \cite{sanders2023engineering} kürzlich einen \textit{generel purpose} Algorithmus entwickelt. Dieser beruht auch auf \boruvkasAlgorithmus und wurde auf über 65.000 Prozessoren evaluiert.

