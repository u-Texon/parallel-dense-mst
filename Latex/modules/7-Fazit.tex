\section{Fazit}\label{Fazit}

Die Experimente haben gezeigt, dass die von Dehne et al. \cite{dehne1998practical} entwickelten Algorithmen in der Praxis auch auf mehreren tausend Prozessoren effizient skalieren. Während \mergeMST selten kürzere Laufzeiten als \boruvkaAllreduce besitzt, so waren \boruvkaMixedMerge und \boruvkaThenMerge auf vielen verschiedenen Eingaben die effizientesten Algorithmen. 
Besonder \boruvkaMixedMerge hat auf dem Pair Graphen, der viele Iterationen von \boruvkaAllreduce benötigt, die höchste Effizienz gezeigt. \\
Zusätzlich konnten wir einen deutlichen Einfluss auf die verteilt parallelen Algorithmen durch lokalen MST Berechnungen beobachten. Nicht nur bildet ein effizienter sequenzieller MST Algorithmus eine wichtige Basis für verteilte MST Algorithmen. Sondern machen diese auch bei sehr dichten Graphen einen Großteil der Laufzeit aus. Dabei war die Verwendung von \textsc{Filter-Kruskal} gegen über Kruskals Algorithmus ein wesentlicher Faktor für eine bessere Effizienz aller Algorithmen. Zusätzlich hat sich \textsc{Filter-Kruskal} als eine sehr effiziente Option offenbart um in einem \boruvkaStep parallele Kanten zu entfernen.\\
Zusätzlich konnten wir beobachten, dass die Überlagerung von Nachrichten ein effizientes Vorgehen sein kann, um einen \boruvkaStep weiter zu beschleunigen.
Allerdings konnten wir aus Zeitgründen dieses Vorgehen nur Anhand einer beispielhaften Implementierung betrachten, die in der Praxis nicht sinnvoll ist.
Außerdem ist eine Nachrichtenüberlagerung nur sinnvoll wenn die lokalen MST Berechnungen einen ähnlich Anteil an der Laufzeit ausmachen wie der Allreduce Aufruf. Dies war bei unseren Experimenten nur auf weniger dichten Graphen möglich, was wiederum die generelle Effizienz der Algorithmen mindert.


Da dieser Ansatz dennoch vielversprechend erscheint, ist eine weiterführende Forschung in diesem Bereich notwendig.
Beispielsweise könnte eine Implementierung mittels Hyper-Threading zeigen, wie effizient dieser Ansatz in der Praxis wirklich ist.
Zusätzlich sollten weitere Experimente folgen, die zeigen welche Eingaben von der Nachrichtenüberlagerung am meisten profitieren.
