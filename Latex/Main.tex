\documentclass[12pt,a4paper,twoside,bibliography=totocnumbered]{scrartcl}


%!!!!!!!!!!!!!!!!!!!!!!!!!!!!!!!!!!!!!!!!!!
% 2 mal farbig Ausdrucken und binden lassen.
% Kapitel Anfang auf Vorderseite (ungerade)
% Per Mail an: sanders@kit.edu, schimek@kit.edu, blancani@kit edu
%!!!!!!!!!!!!!!!!!!!!!!!!!!!!!!!!!!!!!!!!!!
\usepackage[ngerman]{babel}


% Diese (und weitere) Eingabedateien sind in UTF-8
\usepackage[utf8]{inputenc}
\usepackage{csquotes} % provides \enquote{} macro for "quotes"

% Verwende gute Type 1 Font: Latin Modern
\usepackage[T1]{fontenc}
\usepackage{lmodern}

% Hyperref für Hyperlink und Sprungtexte
\usepackage{xcolor,hyperref}
\usepackage{amsthm}
% Einige Standard-Mathematik Pakete
\usepackage{latexsym,amssymb,mathtools,textcomp}
\usepackage[ngerman,noabbrev,nameinlink]{cleveref}

\usepackage[Algorithmus]{algorithm}  % include code snippits
\usepackage{algorithmic}  % include code snippits

\renewcommand{\listalgorithmname}{Algorithmenverzeichnis}

% Seitengröße - verwende fast die ganze A4 Seite
\usepackage[tmargin=22mm,bmargin=22mm,lmargin=20mm,rmargin=20mm]{geometry}

% Einrückung und Abstand zwischen Paragraphen
\setlength\parskip{\smallskipamount}
\setlength\parindent{0pt}



% Unterstützung für Sätze und Definitionen

\newtheorem{thm}{Theorem}

\usepackage[inkscapeformat=png]{svg} %to include svg as images
\usepackage{float} %images

\newtheorem{Satz}{Satz}[section]
\newtheorem{Definition}[Satz]{Definition}
\newtheorem{Lemma}[Satz]{Lemma}

\numberwithin{equation}{section}

\usepackage[
    backend=biber,
    style=numeric,
]{biblatex}

\addbibresource{Bib.bib}

% Unterstützung zum Einbinden von Graphiken
\usepackage{graphicx}

\usepackage[]{mdframed}

% Pakete die tabular und array verbessern
\usepackage{array,multirow}

% Kleiner enumerate und itemize Umgebungen
\usepackage{enumitem}

\setlist[enumerate]{topsep=0pt}
\setlist[itemize]{topsep=0pt}
\setlist[description]{font=\normalfont,topsep=0pt}

\setlist[enumerate,1]{label=(\roman*)}

% TikZ für Graphiken in LaTeX
\usepackage{tikz}
\usetikzlibrary{calc}

% Aktuelle Section und Untersection am Seitenkopf
\usepackage{fancyhdr}

\fancypagestyle{plain}{
  \fancyhead{}
  \fancyfoot{}
  \fancyfoot[LE,RO]{\normalsize\thepage}
  \renewcommand{\headrulewidth}{0pt}
  \renewcommand{\footrulewidth}{0pt}
}

\fancypagestyle{normal}{
  \setlength{\headheight}{20pt}
  \setlength\footskip{32pt}
  \fancyhead{}
  \fancyhead[LE]{\normalsize\textsc{\nouppercase{\leftmark}}}
  \fancyhead[RO]{\normalsize\textsc{\nouppercase{\rightmark}}}
  \fancyfoot{}
  \fancyfoot[LE,RO]{\normalsize\thepage}
  \renewcommand{\headrulewidth}{0.4pt}
  \renewcommand{\footrulewidth}{0pt}
}



\hypersetup{
  pdftitle={Engineering und Evaluation von verteilten MST Algorithmen für dichte Graphen},
  pdfauthor={David Bumm},
  pdfsubject={MST, verteilte Algorithmen},
  colorlinks=true,
  pdfborder={0 0 0},
  bookmarksopen=true,
  bookmarksopenlevel=1,
  bookmarksnumbered=true,
  linkcolor=blue!60!black,
  %linkcolor=black,
  citecolor=blue!60!black,
  urlcolor=blue!60!black,
  filecolor=green!60!black,
  pdfpagemode=UseNone,
  unicode=true,
}

% Paket zum Setzen von Algorithmen in Pseudocode mit kleinen Stilanpassungen
%\usepackage[ruled,vlined,linesnumbered,norelsize]{algorithm2e}
%\DontPrintSemicolon
\def\NlSty#1{\textnormal{\fontsize{8}{10}\selectfont{}#1}}
%\SetKwSty{texttt}
%\SetCommentSty{emph}
\def\listalgorithmcfname{Algorithmenverzeichnis}
\def\algorithmautorefname{Algorithmus}
\let\chapter=\section % repariert ein Problem mit algorithm2e




\begin{document}

%%%%%%%%%%%%%%%%%%%%%%%%%%%%%%%%%%%%%%%%%%%%%%%%%%%%%%%%%%%%%%%%%%%%%%

%\pagestyle{empty} % keine Seitenzahlen

% Titelblatt der Arbeit
\begin{titlepage}

  \begin{center}\large

    \quad\includegraphics[height=17mm]{images/kit_logo_de.pdf} \hfill
    %\includegraphics[height=20mm]{images/grouplogo-algo-blue.pdf}
    \quad\null

    \vfill

    Bachelorarbeit
    \vspace*{2cm}

    {\textbf{\huge Engineering und Evaluation von verteilten MST Algorithmen für dichte Graphen} \par}
    % Siehe auch oben die Felder pdftitle={}
    % mit \par am Ende stimmt der Zeilenabstand

    \vfill

    David Bumm

    \vspace*{15mm}

    01.06.2023 bis 30.10.2023

    \vspace*{45mm}

    \begin{tabular}{rl}
      Betreuer: & M.Sc. Matthias Schimek \\
       Prüfer: & Prof. Dr. Peter Sanders \\
    \end{tabular}
    
    \vspace*{10mm}

    Institut für Theoretische Informatik, Algorithm Engineering \\
    Fakultät für Informatik \\
    Karlsruher Institut für Technologie

    \vspace*{12mm}
  \end{center}

\end{titlepage}

%%%%%%%%%%%%%%%%%%%%%%%%%%%%%%%%%%%%%%%%%%%%%%%%%%%%%%%%%%%%%%%%%%%%%%

\vspace*{0pt}\vfill

\hrule\medskip

Hiermit versichere ich, dass ich diese Arbeit selbständig verfasst und keine anderen, als die angegebenen Quellen und Hilfsmittel benutzt, die wörtlich oder inhaltlich übernommenen Stellen als solche kenntlich gemacht und die Satzung des Karlsruher Instituts für Technologie zur Sicherung guter wissenschaftlicher Praxis in der jeweils gültigen Fassung beachtet habe.

\bigskip

\noindent
Ort, Datum, Name\\

---------------------------------------------------------------------

% Unterschrift (handgeschrieben)

\vspace*{5cm}

\clearpage


%%%%%%%%%%%%%%%%%%%%%%%%%%%%%%%%%%%%%%%%%%%%%%%%%%%%%%%%%%%%%%%%%%%%%%





\vspace*{0pt}\vfill

\begin{abstract}
\centerline{\textbf{Zusammenfassung}}
\hfill


Das \textbf{M}inimum \textbf{S}panning \textbf{T}ree (MST) Problem sucht für einen ungerichteten und (kanten-)gewichteten Eingabegraphen $G = (V, E)$ nach einem Baum $T = (V, E'  \subseteq E)$, welcher alle Knoten aus $V$ verbindet und minimales Gewicht besitzt. \newline
Wir beschäftigen uns mit der Frage, wie MSTs auf Supercomputern mit mehreren tausend Prozessoren effizient berechnet werden können. Solche Supercomputer sind verteilte Systeme, in welchen die Prozessoren nicht über den geteilten Speicher (shared-memory), sondern mittels dedizierter
Nachrichten über Hochleistungsnetzwerke kommunizieren. Dadurch ergeben sich deutlich andere Anforderungen an effiziente Algorithmen als in herkömmlichen shared-memory Systemen.
\newline
Unsere Arbeit beruht auf den Algorithmen von Dehne et al. \cite{dehne1998practical} und Adler et al. \cite{adler1998communication}.\\
Diese Algorithmen arbeiten auf einem verteilt vorliegenden Eingabegraphen, wobei jeder Prozessor $|E|/p$ Kanten erhält. Zusätzlich muss die Knotenmenge auf allen $p$ Prozessoren repliziert werden können. Dafür muss die Anzahl an Kanten um den Faktor $p$ größer sein als die Knotenmenge ($|V| \leq |E| / p$). \\
Auf diesen \enquote{dichten} Graphen hab wir vier verteilte MST Algorithmen implementiert, weitere Verbesserungen für die Praxis vorgenommen und auf über zweitausend Prozessoren evaluiert.


\end{abstract}


\vfill


\vfill\vfill\vfill
\clearpage

\newpage
\
\newpage



%%%%%%%%%%%%%%%%%%%%%%%%%%%%%%%%%%%%%%%%%%%%%%%%%%%%%%%%%%%%%%%%%%%%%%


\pagestyle{normal}
% markiere sections im Seitenkopf links und subsections rechts
\renewcommand\sectionmark[1]{\markboth{\thesection\quad\MakeUppercase{#1}}{\thesection\quad\MakeUppercase{#1}}}
\renewcommand\subsectionmark[1]{\markright{\thesubsection\quad\MakeUppercase{#1}}}

% Inhaltsverzeichnis
\tableofcontents



\newcommand{\mergeStep}{\textsc{MergeStep} }
\newcommand{\boruvkaStep}{\textsc{Bor{\r u}vkaStep} }
\newcommand{\boruvka}{Bor{\r u}vka }
\newcommand{\boruvkasAlgorithmus}{Bor{\r u}vkas Algorithmus }
\newcommand{\boruvkaAllreduce}{\textsc{Bor{\r u}vka-Allreduce} }
\newcommand{\boruvkaAllreduceNoSpace}{\textsc{Bor{\r u}vka-Allreduce}}
\newcommand{\mergeMST}{\textsc{Merge-Local-MST} }
\newcommand{\boruvkaMixedMerge}{\textsc{Bor{\r u}vka-Mixed-Merge} }
\newcommand{\boruvkaMixedMergeNoSpace}{\textsc{Bor{\r u}vka-Mixed-Merge}}
\newcommand{\boruvkaThenMerge}{\textsc{Bor{\r u}vka-Then-Merge} }

\section{Einführung}\label{Einführung}
Das MST Problem zählt zu den fundamentalsten Graphenproblemen überhaupt und bietet Raum für viele algorithmische Ansätze.
Neben der Bedeutung in der Algorithmentheorie ist das MST Problem auch in der Praxis relevant und findet in verschiedenen Bereichen breite Anwendungen. Dazu gehören beispielsweise Clsutering \cite{bateni2017affinity}, Bildsegmentierung \cite{wassenberg2009efficient} und Netzwerkplanung \cite{li2011mst}.

\cref{BasicMST-Img} zeigt \textit{links} einen ungerichteten und gewichteten Beispielgraphen, \textit{mittig} einen beliebigen Spannbaum und \textit{rechts} den dazugehörigen Minimalen Spannbaum.

\begin{figure}[H]
    \centering
    \includesvg[width=16cm]{Figures/Basic-MST.svg}
    \caption{Beispiel Graph mit MST}
    \label{BasicMST-Img}
\end{figure}

\subsection{Motivation}
Graphen geben uns eine universelle und effiziente Möglichkeit verschiedene Sachverhalte für den Computer einfach darzustellen. Für große und komplexe Probleme kann der resultierende Graph enorme Größen annehmen. Um trotzdem noch solche Eingaben schnell genug zu verarbeiten, reichen sequenzielle Algorithmen nicht aus. Durch die parallele Ausführung von mehreren Prozessoren, kann eine wesentliche Beschleunigung erzielt werden. Bei typischen shared-memory Systemen, können alle Prozessoren gemeinsam auf den selben Hauptspeicher lesend oder schreibend zugreifen. Das ermöglicht einerseits eine relativ einfache Implementierung, da jeder Prozessor die gleichen Informationen besitzt. Anderseits bedeutet das für sehr große Eingaben, dass ein einzelner \enquote{gigantischer} Hauptspeicher nötig ist. Dieser muss nicht nur den gesamten Graphen auf einmal speicher können, sonder gegebenenfalls ein vielfaches davon, da bei der Verarbeitung zusätzliche Datenstrukturen oder Ähnliches notwendig sind. \\
Weil so riesige Hauptspeicher in der Realität zu teuer und unrealistisch zu realisieren sind, arbeiten wir auf Supercomputern mit verteilten Speicher. Auf diesen Supercomputern besitzt jeder Prozessor seinen eigenen Speicherbereich, sodass gezielt eine bestimmte Anzahl an Prozessoren für eine Aufgabe ausgewählt werden kann und trotzdem immer gleich viel Speicher für jeden Prozessor zur Verfügung steht.
Das ermöglicht also eine parallele Bearbeitung auf sehr großen Eingaben, indem diese (möglichst) gleichmäßig auf alle Prozessoren aufgeteilt werden. \\
Aber auch für vergleichsweise kleinere Eingaben sind schnelle verteilte Algorithmen sehr relevant. Man kann beispielsweise beobachten, dass viele gut skalierende Simulationen Probleme damit haben, ihre (deutlich kleinere) Datenanalyse effizient auszuwerten. Dies wiederum mindert die eigentlich sehr gute Skalierung.
Manche von diesen Analyse Problemen können als Graphenprobleme umformuliert werden, die in wenigen Millisekunden auf den Daten, die von hoch parallelisierten Maschinen entstehen, gelöst werden müssen. Da in diesem Fall die Daten bereits verteilt vorliegen, eignen sich verteilt parallele Algorithmen für diese Auswertung besonders gut.


\subsection{Problemstellung}
Im Gegensatz zu Systemen mit geteiltem Speicher, wird bei distributed-memory  die Eingabe auf alle Prozessoren verteilt.
Das hat zur Folge, dass jeder Prozessor nur Sicht auf seine eigenen (lokalen) Daten hat und mit anderen Prozessoren kommunizieren muss, um an die gesamten (globale) Informationen zu gelangen. Diese Kommunikationsvorgänge spielen eine erhebliche Rolle bei verteilt parallelen Algorithmen. Besonders in unserem Fall, mit mehreren tausend Prozessoren, kann sowohl die Nachrichtenübertragung als auch notwendige Synchronisationen eine große Auswirkung auf die Laufzeit haben. Die von uns implementierten Algorithmen müssen also diese Aufwände explizit berücksichtigen und ihre Arbeitsweise so gut wie möglich darauf anpassen.\\
Außerdem ist es nicht immer eindeutig wie man den Eingabegraph am besten auf alle Prozessoren aufteilt. Zum Beispiel können auf manchen dünn besetzten Graphen ($|V| \gg |E|$), alle Kanten auf jedem Prozessor gespeichert werden, aber nicht unbedingt alle Knoten. 
Ein verteilter MST Algorithmus muss in diesem Fall unterschiedlich arbeiten als auf anderen Eingaben, weil die lokale Sicht der Daten ein Ausschlaggebender Faktor für das Vorgehen ist.\\
In dieser Arbeit befassen wir uns nur mit einer bestimmten Familie von Graphen, um dieses Problem einheitlich zu betrachten. Im Gegensatz zum eben genannten Beispiel, arbeiten die für uns relevanten Algorithmen nur auf Graphen, bei denen die Knotenmenge auf allen Prozessoren repliziert werden kann. Das bedeutet jeder Prozessor kennt alle Knoten, aber nur $m/p$ Kanten. Das ist der Fall, wenn die Anzahl an Kanten um den Faktor $p$ größer ist als die Knotenmenge. Da in diesem $|V| \leq |E| / p$ ist, also insbesondere $|V| < |E|$ gilt, nennt man den Graphen dicht.
Auf \enquote{kleineren} Eingaben funktionieren die Algorithmen auch mit $|V| \geq |E|$, da auch hier die Knotenmenge auf alle Prozessoren passt, allerdings sind sie hier weniger effizient.

\subsection{Übersicht}
Sequenzielle MST Berechnungen spielen auch in verteilten Algorithmen eine wichtige Rolle für die (lokale) Ausführung auf einzelnen Prozessoren. In \hyperref[Grundlagen]{Kapitel} \ref{Grundlagen} klären wir 
notwendige Grundlagen zu minimalen Spannbäumen und den klassischen sequenziellen MST Algorithmen von Kruskal, \boruvka sowie Jarnik und Prim. Hier stellen wir auch \textsc{Filter-Kruskal}, als eine in der Praxis effizientere Variante von Kruskals Algorithmus vor.\\
Lokalen Berechnungen sind auch wesentlicher Bestandteil für \textsc{Merge-Local-MST}, einer von vier verteilten parallelen Algorithmen, die wir in \hyperref[Algorithms]{Kapitel} \ref{Algorithms} vorstellen. \textsc{Merge-Local-MST} berechnet schrittweise auf jedem Prozessor einen MST und fügt diese anschließend zusammen. 
Da sich Bor{\r u}vkas Algorithmus sehr effektiv parallelisieren lässt, bildet dieser das Herzstück für \boruvkaAllreduce.
Dieser Algorithmus nutzt das Vorgehen von \boruvka, wobei in jeder Iteration ein Allreduce Aufruf durchgeführt wird. Ein Allreduce sammelt von allen Prozessoren Daten, führt sie zusammen und sendet sie anschließend wieder zurück.\\
Um die Vorteile von \boruvkaAllreduce und \mergeMST zu kombinieren haben wir \boruvkaThenMerge und \boruvkaMixedMerge implementiert. Dabei besteht \boruvkaThenMerge aus einer Reiche an Iterationen von \boruvkaAllreduce gefolgt von \mergeMST. \boruvkaThenMerge hingegen, führt abwechselnd \boruvkaAllreduce und \mergeMST aus.\\
Theoretische Laufzeitschranken können in der Praxis nicht immer vorhersagen welcher Algorithmus am besten funktioniert, da dies stark von der Implementierung und den  auftretenden Eingaben abhängt.
Außerdem stimmen die theoretischen Modelle nicht mit den real-existierenden Maschinen überein. Während z.B in theoretischen Analysen die Anzahl an Kanten, Knoten und Prozessoren gegen unendlich gehen, sind die Prozessoren in der Praxis nicht so zahlreich verfügbar. Auch Effekte in der Praxis wie Cache Effizienz werden für theoretische Algorithmen selten in Betracht gezogen, obwohl diese massive Unterschiede erzielen können.
In \hyperref[Implementierung]{Kapitel} \ref{Implementierung} nennen wesentlichen Faktoren unserer Implementierung so wie auf welchen Eingaben wir unsere Algorithmen evaluieren.\\
Eine praktische Evaluation mit bis zu 2048 Prozessoren zu den verwendeten Algorithmen ist in \hyperref[Evaluierung]{Kapitel} \ref{Evaluierung} aufgeführt.
Hierbei vergleichen wir einerseits die jeweiligen Laufzeiten auf drei verschiedenen Graphtypen. Andererseit zeigen wir die Auswirkung von lokalen Berechnungen, mit welchen Parametern die Algorithmen am effizientesten sind und nutzen die Zeit für die Nachrichtenübertragung aus um währenddessen weitere Berechnungen durchzuführen. \\
In \hyperref[Fazit]{Kapitel} \ref{Fazit} besprechen wir die gesammelten Ergebnisse und erörtern weitere Verbesserungen für mögliche weitere Arbeiten.





\newpage
\
\newpage
\chapter{Theoretische Grundlagen}\label{Grundlagen}


\subsection{Minimaler Spannbaum (MST)}
In der Graphentheorie versteht man unter einem Graph $G = (V,E)$ eine Menge von Knoten (\textbf{V}ertices) und Kanten (\textbf{E}dges), die eine Verbindung zwischen zwei Knoten darstellen. Insbesondere gibt $|V|$ oder auch $n$ die Anzahl an Knoten an und $|E|$ bzw. $m$ die Anzahl an Kanten. Zusätzlich können Kanten noch ein Gewicht (oder Kosten) enthalten. 
%Damit hat eine Kante die Darstellung $(s,t,w)$ wobei $s$ der Urspungsknoten, $t$ der Zielknoten und $w$ das Gewicht der Kante beschreibt.
Im Allgemeinen sind Kanten bei dem MST Problem ungerichtet, das bedeutet, dass jede Kante $\{s,t\}$ sowohl von Knoten $s$ nach $t$, als auch von $t$ nach $s$ durchlaufen werden kann. Wir gehen im Folgenden immer von ungerichteten Graphen als Eingabe aus.
%Wir nennen einen Graph dicht, wenn er deutlich mehr Kanten als Knoten besitzt, also wenn $m \gg n$ gilt.\\

Ein minimaler Spannbaum (engl. \textbf{M}inimum \textbf{S}panning \textbf{T}ree oder kurz MST) ist die Teilmenge eines Graphen, bei dem alle Knoten miteinander verbunden sind und die Summe aller Kantengewichte minimal ist. Besteht der Eingabegraph aus einer einzigen Komponente, so ist das Ergebnis einer MST Berechnung ein einzelner Baum. Also ein Graph mit genau $n-1$ Kanten, bei dem alle Knoten über einen eindeutigen Pfad aus Kanten verbunden sind.
Ist der Graph nicht zusammenhängend, also besteht er aus mehr als einer Komponente, so kann iterativ auf jeder Komponente eine MST Berechnung durchgeführt werden. Das Ergebnis ist in diesem Fall ein minimaler Spannwald (engl. \textbf{M}inimum \textbf{S}panning \textbf{F}orest).
Um also immer einen MST als Ergebnis zu erhalten, können wie ohne Beschränkung der Allgemeinheit annehmen, dass der Eingabegraph aus genau einer Komponente besteht.\\
Der MST eines Graphen ist nicht immer eindeutig. Hat beispielsweise jede Kante in einem Graph Gewicht 1, so kann es eine Vielzahl an unterschiedlichen MSTs geben, wenn es mehrere Möglichkeiten gibt die Knoten miteinander zu verbinden. Sind die Kantengewichte allerdings eindeutig, also kommt jedes Kantengewicht im Graphen höchstens ein mal vor, so ist der resultierende MST auch eindeutig \cite{sanders2019sequential}. 


\subsection{Schnitt- und Kreiseigenschaft}\label{Eigenschaften}

Die Schnitteigenschaft, zusammen mit der Kreiseigenschaft, bilden wichtige Merkmale für die Auswahl von MST Kanten in einem Graph. Anhand dieser Eigenschaften konnte die Korrektheit von verschiedenen MST Algorithmen bewiesen werden.
In \hyperref[Korrektheit]{Kapitel} \ref{Korrektheit} nutzen wie die Kreiseigenschaft um die Korrektheit eines verteilten Algorithmus zu begründen.

\begin{mdframed}
\begin{thm}\label{Schnitteigenschaft}
    Schnitteigenschaft
\end{thm}
Sei $S_1,S_2$ $\subset$ V mit $S_1\cup S_2 = V$ und $S_1\cap S_2 = \emptyset$. Die (Schnitt-) Menge an Kanten, die zwischen $S_1$ und $S_2$ verlaufen, nennen wir $E_s$.
In diesem Fall besagt die Schnitteigenschaft, dass die Kante $e\in E_s$ mit dem geringsten Gewicht aus $E_s$ immer Teil des MSTs von G ist.
\end{mdframed}


In \cref{Schnitt-Img} sind die Teilmengen $S_1=\{A, D, E\}$ und $S_2=\{B, C, F\}$, sowie die zugehörige Schnittmenge $E_s=\{(A,B,7), (B,E,6), (C,E,5) (E,F,3)\}$ zu sehen. Nach der Schnitteigenschaft gehört die Kante $e =(E,F,3)$ auf jeden Fall zum MST von G, da $e$ minimales Gewicht in $E_s$ hat.


\begin{figure}[H]
    \centering
    \includesvg[width=8cm]{Figures/Schnitt.svg}
    \caption{Beispiel für die Schnitteigenschaft}
    \label{Schnitt-Img}
\end{figure}



\begin{mdframed}
\begin{thm} \label{Kreiseigenschaft}
    Kreiseigenschaft
\end{thm}
Sei $K \subseteq E$ ein beliebiger Kreis des Graphen $G$. Dann besagt die Kreiseigenschaft, dass die Kante $e \in K$ mit dem höchsten Gewicht aus K nicht im MST von G enthalten ist.
\end{mdframed}

Wir sehen in \cref{Kreis-Img} einen Graphen mit eingefärbten Kreis $K =\{(A,B,7), (B,C,2), (C,E,5), \\ (D,E,4), (A,D,1)\}$.
Wir wissen also nun, dass die Kante $(A,B,7)$ nicht zum MST gehören kann, da sie die Kante mit dem höchsten Gewicht in $K$ ist.

\begin{figure}[H]
    \centering
    \includesvg[width=6cm]{Figures/Kreis.svg}
    \caption{Beispiel für die Kreiseigenschaft}
    \label{Kreis-Img}
\end{figure}


Einen Beweis für die Kreis- und Schnitteigenschaft findet man beispielsweise im Buch von Sanders et al. \cite{sanders2019sequential}.



\subsection{Sequenzielle Algorithmen}
Auch wenn wir uns im Folgenden nur mit verteilt parallelen Algorithmen auseinander setzen, spielen auch lokale MST Berechnungen bei diesen eine wichtige Rolle.\\
Die Algorithmen von Kruskal \cite{kruskal1956shortest}, Jarník und Prim \cite{prim1957shortest} sowie \boruvka \cite{boruuvka1926jistem} bilden die bekanntesten sequenziellen MST Algorithmen.

Bor{\r u}vkas Algorithmus wurde in 1926 veröffentlicht und ist damit der älteste MST Algorithmus. Dieser fügt in jeder Iteration die leichtesten inzidenten Kanten aller Knoten zum MST hinzu und kontrahiert anschließend den Graph, bis nur noch ein Knoten übrig ist. Die Laufzeit liegt dabei in $O(n \log(m)$.
Ein ausschlaggebender Vorteil von Bor{\r u}vkas Algorithmus ist, dass man ihn sehr einfach und effektiv parallelisieren kann.\\

Der Jarník-Prim Algorithmus wurde 1930 von Jarník entwickelt und 1957 von Prim wieder entdeckt. Er nutzt die Schnitteigenschaft (\hyperref[Schnitteigenschaft]{Theorem} \ref{Schnitteigenschaft}) aus um den MST zu berechnen. So startet der Algorithmus mit einem beliebigen Knoten von G und fügt in jeder Iteration, die leichteste Schnittkante und zwischen den bisher hinzugefügten Knoten und dem restlichen Graph hinzu. So wird nach der Schnitteigenschaft in jedem Schritt eine MST-Kante und ein Knoten hinzugefügt, solange bis alle Knoten hinzugefügt wurden. Unter Verwendungen einer geeigneten Prioritätswarteschlange (z.B Fibonacci Heaps) liegt die Laufzeit von Jarník und Prims Algorithmus in $O(m+n\log n)$.

1956 hat Kruskal seinen Algorithmus entwickelt welcher nach und nach die leichteste Kante des Graphen zum MST hinzufügt, sofern sie keinen Kreis im Graphen schließt. 
Mit einer Laufzeit von $O(m\log m)$ ist Kruskal auf ausreichend dichten Graphen asymptotisch ineffizienter als Jarník-Prim oder Bor{\r u}vka. 
Allerdings zeigen Osipov et. al \cite{osipov2009filter}, dass (Filter-)Kruskal in der Praxis für viele Graphinstanzen effizienter ist als Jarník-Prim. 
Zusätzlich ist Kruskal auch simpler zu Implementieren und aus diesen beiden Gründen nutzen wir im folgenden ausschließlich Kruskal (bzw. Filter-Kruskal) für unsere lokalen MST Berechnungen.

Bor{\r u}vkas Algorithmus bildet in \hyperref[Algorithms]{Kapitel} \ref{Algorithms} eine wichtige Basis für die verteilten Algorithmen, wobei Kruskal für die lokalen MST Berechnungen am effizientesten ist. Da uns Jarník-Prim keine ausschlaggebenden Vorteile bietet, betrachten wir diesen nicht weiter und gehen im Folgenden genauer auf das Vorgehen von \boruvka und Kruskal ein.

\subsubsection{Bor{\r u}vkas Algorithmus}
Zu Beginn jeder Iteration von Bor{\r u}vkas Algorithmus wird für jeden Knoten, die inzidente Kante mit dem geringsten Gewicht zum MST hinzugefügt.
Als nächstes werden alle Knoten, die über diese leichtesten inzidenten Kanten verbunden sind, kontrahiert und die noch übrigen Kanten entsprechend umbenannt. \\
Bei der Umbenennung entstehen dann gegebenenfalls mehrmals die gleichen (parallele) Kanten, von denen man alle bis auf die leichteste entfernen kann. Das entfernen der parallelen Kanten ist hierbei für die Korrektheit des Algorithmus nicht zwangsläufig nötig. \\
Dieses Vorgehen wiederholen wir so oft, bis nur noch ein Graph mit einem Knoten übrig ist. \\
Da bei einer Kontraktion des Graphen, immer mindestens zwei Knoten entlang einer Kante zusammengefügt werden, verringert sich in jeder Iteration, die Anzahl an Knoten um mindestens die Hälfte. Damit 
sind höchstens $\log(n)$ Iterationen notwendig um den MST zu berechnen. In jeder Iteration werden alle $m$ Kanten einmal durchlaufen um die leichtesten inzidenten Kanten zu finde und ein weiteres mal um diese umzubenennen. Die Kontraktion des Graphen ist in $O(n) \subseteq O(m)$ möglich. Die gesamte Laufzeit von Bor{\r u}vkas sequenziellen Algorithmus liegt damit in $O(m\log(n))$.\\
Aufgrund der Umbenennung der Kanten muss man allerdings zum Schluss diese wieder in ihre Ursprungsform bringen, sofern man nicht nur am Gewicht interessiert ist.\\

Der Pseuocode zu einer parallelen Variante von \boruvka findet sich bei \cref{Boruvka-Allreduce-Algo}. In \cref{Boruvka-Img} ist ein kompletter Durchlauf als Beispiel zu sehen. Hier werden in der ersten Iteration von \boruvka die Kanten $(A,D,1)$, $(B,C,2)$ und $(E,F,3)$ zum MST hinzugefügt und die jeweiligen Knoten zusammengefasst. Damit sind im nächsten Schritt nur noch 3 Knoten übrig, wobei hier die Kanten $(AD,EF,4)$ und $(BC,EF,5)$ zum MST gehören. Anschließend können alle Knoten zu einem kontrahiert werden, der Algorithmus terminiert und liefert den MST mit Gewicht 15.

\begin{figure}[H]
    \centering
    \includesvg[width=16cm]{Figures/Boruvka.svg}
    \caption{Beispieldurchlauf von Bor{\r u}vkas Algorithmus}
    \label{Boruvka-Img}
\end{figure}


\subsubsection{Kruskals Algorithmus}
Als erstes sortiert Kruskals Algorithmus
alle Kanten aufsteigend nach Kantengewicht. Anschließend wird für jede Kante ${s,t}$ überprüft ob die Knoten $s$ und $t$ bereits über andere Kanten verbunden sind. Ist das der Fall, so schließt diese Kante einen Kreis, da $s$ und $t$ bereits über einen anderen Pfad verbunden sind. Wegen der Sortierung ist diese Kante im Kreis diejenige mit dem größten Gewicht und kann nach der \hyperref[Kreiseigenschaft]{Kreiseigenschaft} verworfen werden.
Andernfalls wird die Kante zum MST hinzugefügt.\\
Um effizient zu überprüfen ob $s$ und $t$ bereits in der selben Komponente liegen, wird häufig die UnionFind Datenstruktur verwendet. Diese Datenstruktur verfügt über die Operation \emph{union(s,t)} und \emph{find(s)}, welche zwei Knoten zur selben Zusammenhangskomponente hinzufügt (union) oder überprüft (find) in welcher Zusammenhangskomponente ein Knoten bereits enthalten ist. Eine union bzw. find Operation benötigt $O(\log(n))$ Operationen, so dass der Kruskal Algorithmus $O(m(\log(m))$ Operation zum Sortieren, gefolgt von $O(n)$ union und $O(m)$ find Operationen benötigt.
Insgesamt hat Kruskals Algorithmus eine amortisierte Laufzeit von $O(m \alpha(m,n))$, wobei $\alpha$ die inverse Ackermann-Funktion ist.
Der folgende Pseudocode von \cref{Kruskal-Algo} zeigt das Vorgehen von Kruskal.

\begin{algorithm} 
\caption{Kruskal(V, E, UF: UnionFind): Kantenliste}
\begin{algorithmic}[1]
\label{Kruskal-Algo}

\STATE MST: Kantenliste
\STATE $E_{sorted}$ $\gets$ sortiere E aufsteigend nach Kantengewicht
\FOR{\textbf{each} $e=(s,t,w) \in E_{sorted}$ }
    \IF{UF.find(e.s) $\neq$ UF.find(e.t)} 
        \STATE MST $\gets$ MST $\cup \{e\}$ 
        \STATE UF.union(e.s, e.t)
    \ENDIF
\ENDFOR

\RETURN MST

\end{algorithmic}
\end{algorithm}

\newpage


\subsection{Filter-Kruskal}
Der \textsc{Filter-Kruskal} Algorithmus \cite{osipov2009filter} ist eine Abwandlung von Kruskals Algorithmus und dient ebenso zur sequenziellen Berechnung von MSTs. Die Idee ist es den Hauptaufwand von Kruskal, dem Sortieren der Kanten, in der Praxis zu verbessern.\\
Hierfür wird ein Ansatz ähnlich zum \emph{quick sort Algorithmus} \cite{hoare1962quicksort} verwendet: \\
Zunächst wird eine zufällige Kante als Pivot ausgewählt und die zu sortierende Kantenmenge in zwei Mengen $E_{\leq}$ und $E_>$ aufgeteilt. Wobei in $E_>$ alle Kanten mit einem größeren Gewicht, als die Pivotkante enthalten sind und in $E_{\leq}$ die übrigen Kanten.
Ist $E_{\leq}$ klein genug, so wird auf dieser Menge der normale Kruskal ausgeführt, ansonsten wird erneut eine $E_{\leq}$ und $E_>$ Menge gebildet. Auf $E_>$ führen wir anschließend einen filter-Schritt durch. Dabei wird mithilfe der UnionFind Datenstruktur überprüft, ob Kanten aus dieser Menge bereits nicht mehr benötigt werden. Schließlich wird auch diese Menge (sofern nötig) erneut in $E_{\leq}$ und $E_>$ aufgeteilt. Der Wert für die Grenze, die angibt ab wann Kruskals Algorithmus ausgeführt wird, kann variieren, sollte aber in $O(n)$ liegen \cite{osipov2009filter}.
Der Pseudocoe von \cref{Filter-Kruskal-Algo} zeigt dieses Vorgehen im Detail.\\



\begin{algorithm} 
\caption{\textsc{Filter-Kruskal}(V, E, UF: UnionFind, Grenze: int): Kantenliste}
\begin{algorithmic}[1]
\label{Filter-Kruskal-Algo}

\IF{|E| < Grenze}
    \RETURN Kruskal(V, E, UF)
\ENDIF

\STATE pivotGewicht $\gets$ Gewicht einer zufälligen Kante $e \in E$
\STATE $E_{\leq}$ $\gets$ e $\in$ E mit e.w $\leq$ pivotGewicht
\STATE $E_>$ $\gets$ e $\in$ E mit e.w > pivotGewicht
\STATE $E_{\leq}$ $\gets$ Filter-Kruskal(V, $E_{\leq}$, UF, Grenze)
\STATE $E_>$ $\gets$ Filter($E_>$, UF)
\STATE $E_>$ $\gets$ Filter-Kruskal(V, $E_>$, UF, Grenze)

\RETURN  $E_{\leq}$ $\cup$ $E_>$


\end{algorithmic}
\end{algorithm}


\begin{algorithm} 
\caption{\textsc{Filter}(E, UF: UnionFind): Kantenliste}
\begin{algorithmic}[1]
\label{FilterStep-Algo}

\RETURN e $\in$ E mit (UF.find(e.s) != UF.find(e.t))

\end{algorithmic}
\end{algorithm}



Die erhöhte Effizienz in der Praxis stammt daher, dass in vielen MSTs nur \emph{leichte} Kanten enthalten sind. In diesem Fall werden \enquote{schweren} Kanten nicht mit sortiert, sonder über den \textsc{Filter} Aufruf vorher entfernt. Die asymptotische Laufzeit ist also identisch zu Kruskal, aber auf zufälligen Graphen liegt die erwartete Laufzeit in 
$O(m + n\log n \log \frac{m}{n})$ \cite{osipov2009filter}.





\subsection{Parallele Modelle}
Es gibt verschiedene theoretische (Berechnungs-) Modelle um die Operationen (und Komplexität) von parallelen Algorithmen zu beschreiben. Da wir uns  mit verteilten Algorithmen beschäftigen, sind Modelle, die für geteilten Speicher entwickelt wurden, wie PRAM (\textbf{p}arallel \textbf{R}andom \textbf{A}ccess \textbf{M}achines) für uns nicht ausreichend. Wir wollen die Komplexität von
Nachrichtenübertragungen explizit mit berücksichtigen und betrachten daher im Folgenden das BSP und $\alpha$/$\beta$ Modell.


\subsubsection{Das Bulk Synchronous Parallel Modell}
1990 wurde das BSP Modell von Valiant et al. \cite{valiant1990bridging} entwickelt um die Kommunikation und Synchronisation von verteilten Algorithmen berücksichtigen zu können. 
In diesem Modell gibt es eine Maschine mit $p$ Prozessoren, wobei jeder Prozessor seinen eigenen Speicherbereich besitzt. Zusätzlich gibt es einen Router mit Kommunikationsdurchatz $g$. Eine Synchronisation kann alle $L$ Zeitschritte stattfinden. Ein BSP Algorithmus besteht aus einer Reihe von sogenannten \textit{Supersteps}, welche von Barrieren-Synchronisationen getrennt sind. Insofern können Nachrichten, die in Superstep i gesendet werden, erst im folgenden Superstep i+1 für die Berechnung verwendet werden.
Jeder Superstep i hat einen Aufwand von $w_i+gh_i+L$. Dabei ist im i-ten Superstep $w_i$ die größte Anzahl an lokalen Operationen, und $h_i$ an gesendeten oder empfangenen Nachrichten, aller p Prozessoren.
Die gesamten Kosten eines Algorithmus sind als $W+gH+LT$ angegeben, wobei $W=\sum_i w_i$, $H=\sum_i h_i$ und $T$ die Anzahl an Supersteps ist.


Dieses Modell haben Dehne et al. \cite{dehne1998practical} und Adler et al. \cite{adler1998communication} verwendet um die Komplexität der uns zugrunde legenden Algorithmen anzugeben. Wir werden die Algorithmen zusätzlich im fein-körnigeren Alpha-Beta Modell evaluieren um einen zusätzlichen Einblick in die jeweiligen Laufzeiten zu erhalten.






\subsubsection{Das \boldmath$\alpha$/$\beta$ Modell}
Im Gegensatz zum BSP Modell werden im $\alpha$/$\beta$ (Alpha/Beta) Modell alle Nachrichtenübertragung zwischen \textit{processing elements} (PEs) einzeln Berücksichtigt \cite{sanders2019sequential}. Das ermöglicht eine detailliertere Analyse für die Laufzeit und Komplexität verteilter Algorithmen.
Die Übertragung einer Nachricht der Länge $\ell$ zwischen zwei PEs benötigt $\alpha + \beta \ell$ Zeit. Wobei $\alpha$ die Zeit für die Initialisierung (Startup) der Übertragung ist und $\beta$ die Zeit zum senden einer Dateneinheit angibt.


\subsection{Kollektive Operationen}
Kollektive Operationen sind Bestimme Aufrufe, die für die Kommunikation zwischen PEs verwendet werden.
Sie ermöglichen einen einfachen und effizienten Umgang für den Nachrichtenaustausch von zwei PEs, bis hin zu allen auf einmal. Da kollektive Operationen für die Algorithmen ein wichtiger Bestandteil sind, geben wir nun eine kurze Übersicht über deren Funktionsweise und Komplexität. 

\subsubsection{Boradcast}
Der \emph{Boradcast} Aufruf wird verwendet, wenn ein PE eine Nachricht der Länge $\ell$ mit alle anderen PEs teilen will. Die untere Schranke für die Laufzeit ist dabei $\alpha + \log(p) + \beta \ell$ \cite{sanders2009two}. 
Die Laufzeit ergibt sich logarithmisch in der Anzahl $p$ an PEs, da sich die Anzahl an PEs, die die Nachricht weiter senden kann, in jedem Schritt verdoppelt.

\subsubsection{(All-)Reduce}
Angenommen jeder PE i besitzt einen Vektor $M_i$ der Länge $\ell$ von Typ $T$. Sei zusätzlich $\bigoplus$ eine assoziative binäre Operation auf dem Datentyp $T$.
Der \emph{Reduce} Aufruf führt alle Daten auf einem PE zusammen und berechnet $M := \bigoplus_{i=0}^{p-1} M_i$. Ein \textbf{All}reduce hingegen funktioniert  wie ein Reduce gefoltg von einem Boradcast Aufruf.
Das bedeutet nach dem Allreduce Aufruf liegt  $M$ anschließend nicht nur auf einem PE, sonder auf allen PE's vor. Dabei liegt die Laufzeit von einer Reduce als auch Allreduce Operation in $O(\alpha \log(p) + \beta \ell)$ \cite{sanders2009two}.

\section{Verwandte Arbeiten}\label{RelatedWork}

Da das MST Problem eines der grundlegendsten Graphenprobleme ist, wird sich schon lange damit beschäftigt, effiziente Algorithmen für dieses Problem zu finden. Seitdem die bereits erwähnten Algorithmen von Kruskal \cite{kruskal1956shortest}, Jarnik und Prim \cite{prim1957shortest} sowie \boruvka \cite{boruuvka1926jistem} entwickelt wurden, gibt es viele Arbeiten, die darauf aufbauen.

So hat zum Beispiel in 1995 Karger et al. \cite{karger1995randomized} den KKT-Algorithmus entworfen. Dieser kombiniert \boruvka zusammen mit randomisierten Stichproben und effektivem Kanten filtern, sodass dieser eine erwartete lineare Laufzeit erreicht hat.
Ansonsten hat Osipov et al. \cite{osipov2009filter} in 2009 den Filter-Kruskal Algorithmus entwickelt. 
Dieser Algorithmus verbessert in der Praxis den hohen Aufwand des Kanten Sortierens bei Kruskal, weshalb wir ihn auch für unsere lokale Berechnungen weiter verwenden. 

Natürlich gibt es mittlerweile auch eine Vielzahl an parallelen MST Algorithmen. In 2021 hat Dhulipala \cite{dhulipala2021theoretically} eine shared-memory Varainte von \boruvkasAlgorithmus entwickelt. Wobei erst letztes Jahr Esfahani et al. \cite{koohi2022mastiff} einen \textit{structure aware} Algorithmus veröffentlicht hat, der diesen noch übertrifft.

Neben diesen shared-memory Varianten gibt es noch weitere parallele Algorithmen, die wie wir auf geteiltem Speicher arbeiten. So haben Chung et al. \cite{chung1996parallel} in 1996 erstmalige eine verteilte Versions von \boruvkasAlgorithmus entwickelt. Nur zwei Jahre später veröffentlichten Dehne und Götz \cite{dehne1998practical} ihre Algorithmen, auf denen wir mit unserer Arbeit aufbauen.
Im selben Jahr veröffentlichten auch Adler et al. \cite{adler1998communication} eine ähnliche Reihe an Kommunikationsoptimalen Algorithmen.

Im Gegensatz zu diesen Algorithmen, die für dichte Graphen konstruiert, wurden haben Sanders und Schimek \cite{sanders2023engineering} kürzlich einen \textit{generel purpose} Algorithmus entwickelt. Dieser beruht auch auf \boruvkasAlgorithmus und wurde auf über 65.000 Prozessoren evaluiert.


\newpage
\
\newpage
\section{Algorithmen}\label{Algorithms}
In diesem Kapitel betrachten wir die Funktionsweise, Eigenschaften und Komplexität der von uns implementierten verteilt parallelen Algorithmen \cite{adler1998communication, dehne1998practical}.
Wir gehen ab sofort davon aus, dass der Graph bereits verteilt auf den PEs vorliegt, welche von $0$ bis $p-1$ durchnummeriert sind.
Damit ist die lokale Kantenmenge $E_{\ell i}$ je nach PE i unterschiedlich ($\bigcap_{i=0}^{p-1} E_{\ell i}=\emptyset$) und es gilt für die globale Kantenmenge $E = \bigcup_{i=0}^{p-1} E_{\ell i}$.
Am Ende jedes Algorithmus liegt der globale MST vollständig auf dem PE mit Identifikator (ID) 0 vor.

\subsection{\mergeMST}
Die Idee des \mergeMST Algorithmus ist es schrittweise auf jedem PE einen lokalen MST zu berechnen und diesen anschließend mit dem MST von anderen PEs zusammen zu fassen. Dieser Vorgang wird solange wiederholt bis noch genau ein PE mit dem globalen MST übrig ist.


\subsubsection{Funktionsweise}
In jeder Iteration i des  \textsc{Merge-Local-MST} Algorithmus berechnen alle noch aktiven PEs auf ihren lokalen Kanten einen MST (z.b mittels Kruskal). Anschließend empfängt jeder $D$-te aktive PE die MSTs von den folgenden aktiven $D-1$ PEs. Nachdem ein PE seinen MST versendet hat, ist dieser anschließend nicht mehr aktiv und wird nicht weiter benötigt. Die restlichen PEs können aus den empfangen Daten nun erneut einen lokalen MST berechnen.
Dieses Vorgehen wird solange wiederholt bis nur noch ein Prozessor aktiv ist. Dieser berechnet nun ein letztes Mal einen MST und gibt diesen zurück.

Wie viele MSTs in jedem Schritt zusammengefasst werden bezeichnen wir als \emph{treefacor} oder auch $D$. Hierbei kann $D$ (kleiner als p) beliebig gewählt werden. So kann es vorkommen, dass in einer Iteration ein PE weniger als $D-1$ MSTs empfängt, falls $\log_D(p) \notin \mathbb{N}$.\\
Insgesamt sind also $\lfloor \log_{D}($p$) \rfloor $ Iterationen nötig.
Eine Iteration dieses Verfahrens ist in \cref{MergeStep} geschildert und das vollständige Vorgehen in \cref{Merge-Local-MST-Algo}.


In \hyperref[eval-treefactor]{Kapitel} \ref{eval-treefactor} arbeiten wir heraus, welche Werte sich für $D$ in der Praxis eignen.\\
\cref{MergeMST-Img} zeigt wie das Zusammenfügen der MSTs für 8 PEs mit einem treefactor von 2 funktioniert. 
Zu beginn sind alle acht PEs aktiv und in jeder Iteration sendet ein PE seinen MST weiter, sodass immer $D=2$ MSTs verschmolzen werden. So sind z.B nach der ersten Iteration nur noch die PEs 0, 2, 4 und 6 aktiv und insgesamt nach $\log_2(8) = 3$ Iterationen liegt der globale MST auf PE 0 vor.\\

\newpage
 

\begin{figure}[H]
    \centering
    \includesvg[width=11cm]{Figures/Merge.svg}
    \caption{Übersicht zur Verschmelzung von MSTs}
    \label{MergeMST-Img}
\end{figure}


\begin{algorithm} 
\caption{\textsc{MergeStep}(V, MST: Kantenliste, D: int)}
\begin{algorithmic}[1]
\label{MergeStep}
\STATE andereMSTs: Liste an Kantenlinsten
\IF{(PE ist empfänger)}
    \STATE andereMSTs $\gets$ empfange (biszu) $D$ MST 
\ELSE
    \STATE sende MST zu Empfänger PE
    \RETURN //Dieser PE ist nun inaktiv
\ENDIF
\FOR{Kanten $\in$ andereMSTs}
    \STATE MST $\gets$ MST $\cup$ Kanten
\ENDFOR
\STATE MST $\leftarrow$ localMST(MST, V)
\end{algorithmic}
\end{algorithm}

\begin{algorithm} 
\caption{\textsc{Merge-Local-MST}(V, E, D: int): Kantenliste}
\begin{algorithmic}[1]
\label{Merge-Local-MST-Algo}
 
\STATE MST $\leftarrow$ localMST(E, V)
\FOR{i $\leftarrow$ 0 bis $\lfloor \log_{D}($p$) \rfloor $}
    \STATE MergeStep(V, MST, D)
\ENDFOR

\RETURN MST
\end{algorithmic}
\end{algorithm}



\subsubsection{Eigenschaften und Komplexität}
Sei $m'$ die maximale Anzahl an lokalen Kanten eines PEs und $T_{seq}$ die Laufzeit für eine lokale MST Berechnung.
%Dann liegt die Laufzeit im BSP Modell für eine Iteration von Merge-Local-MST bei $W+gh+LT$, mit $W=T_{seq}(n,Dm') + O(Dm'), H\leq Dm'$\ und $T=1$. \\
Dann liegen im Alpha-Beta Modell die Kosten für das Senden eines MSTs $\alpha + \beta \ell$, wobei $\ell$ die Länge des zu sendenden MSTs ist.  Angenommen $m \geq n$, so ergibt sich die Laufzeit einer Iteration durch eine lokale MST Berechnung, gefolgt von $D$ Nachrichtenübertragungen für das Empfangen der MSTs mit einer jeweiligen maximalen Länge von $n-1$. In diesem Fall liegt die Laufzeit für den ersten \textsc{MergeStep} in  $O(T_{seq}(n,m') + D(\alpha +\beta n))$. 
Im Anschluss hält jeder aktive PE höchstens n-1 Kanten. Daraus ergibt sich die Gesamte Laufzeit von \textsc{Merge-Local-Mst} in 
\begin{center}
    $O(T_{seq}(n,m') +
    \log_{D}(p) (T_{seq}(n,Dn) + D(\alpha +\beta n)))$.
\end{center}

Sei nach wie vor $m \geq n$, dann können wir ohne Beschränkung der Allgemeinheit annehmen, dass jeder PE nach Iteration i einen MST mit höchstens n-1 Kanten hält. Dann reduziert sich, in einem weiteren  \textsc{MergeStep},die Anzahl der übrigen Kanten um den Faktor $D$.

Ein Nachteil dieses Algorithmus ist natürlich, dass nur im ersten Schritt wirklich alle PEs genutzt werden und bereits in der nächsten Iteration mindestens die Hälfte inaktiv sind.
Außerdem ist \textsc{Merge-Local-MST} auf dünn besetzten Graphen deutlich weniger effektiv. In diesem Fall werden bei einer lokalen MST Berechnung fast alle Kanten beibehalten, da bei $m \ll n$ beinahe alle Kanten zu einem MST dazugehören.

\subsubsection{Korrektheit von lokalen MST Berechnungen}\label{Korrektheit}
\mergeMST liefert nur einen korrekten MST, wenn bei lokalen MST Berechnungen in keinem Fall eine MST Kanten aus dem globalen Graph entfernt wird.
Tatsächlich ist diese Aussage korrekt und wir können lokale MST Berechnungen auch bei den folgenden Algorithmen verwenden. \\
Angenommen jeder PE i hat auf seinen lokalen Kanten $E_{\ell i}$ einen MST berechnet und dabei die Kanten $E_{Ri}$ entfernt. Das bedeutet, dass für eine Kante $\{s,t\} \in E_{Ri}$, die Knoten $s$ und $t$ bereits lokal über einen günstigeren Pfad verbunden sind. Global betrachtet, kann höchstens noch besserer Pfad von $s$ nach $t$ gefunden werden, aber die Kante $\{s,t\}$ wird nie eine Verbesserung dafür sein. Angenommen wir fügen die Kante $\{s,t\}$ dennoch zum MST hinzu, so schließt sich ein Kreis $K$ und die Kante $e \in K$  mit dem größten Gewicht kann nach der \hyperref[Kreiseigenschaft]{Kreiseigenschaft} verworfen werden. Das größte Gewicht in $K$ muss aber die Kante $\{s,t\}$ haben, da sie sonst bereits in der lokalen MST Berechnung zum MST gehört hätte.

\subsection{\boruvkaAllreduce}
Der \boruvkaAllreduce Algorithmus ist eine verteilt parallele Variante von \boruvkasAlgorithmus \space\cite{boruuvka1926jistem}. \\
Jeder PE rechnet die lokalen leichtesten inzidenten Kanten aus und verständigt sich mittels eines Allreduce Aufrufs, um die globalen Kanten zu erhalten. Anschließend kontrahiert jeder PE den Graphen wie im sequenziellen \boruvka Algorithmus und führt die Berechnung fort. Der Pseudocode von \cref{Boruvka-Allreduce-Algo} veranschaulicht dieses Vorgehen zusammen mit der folgenden Beschreibung der  \hyperref[Funktion-Borruvka]{Funktionsweise}.

\subsection{Funktionsweise}\label{Funktion-Borruvka}
Am Anfang einer Iteration kann jeder PE $p_j$ auf seiner lokalen Kantenmenge einen MST berechnen. Dieser Schritt ist optional, kann sich aber positiv auf die Laufzeit auswirken (siehe \hyperref[remove-Section]{Kapitel} \ref{remove-Section}).
Als nächstes berechnet $p_j$ für jeden Knoten diejenige inzidente Kante mit dem geringsten Gewicht und schreibt diese in ein Array $I_L$ der Länge n (\hyperref[Incident-start]{Zeile} \ref{Incident-start} - \ref{Incident-end}). Dabei steht im i-ten Eintrag von $I_L$, die leichteste inzidente Kante zu Knoten i. Um das globalen inzidente Kanten Array $E_I$ zu erhalten, nutzen wir die Allreduce Operation, welche das Array $I_L$ von jedem Prozessor als Eingabe bekommt (\hyperref[Allreduce-code]{Zeile} \ref{Allreduce-code}). Hier werden alle Arrays $I_L$ miteinander verglichen, sodass jeder Prozessor ein Array $E_I$ mit den global kleinsten inzidenten Kanten als Ausgabe erhält. \\
Diese Kanten fügt nur PE $p_0$ zum MST hinzu (\hyperref[addMST-start]{Zeile} \ref{addMST-start} - \ref{addMST-end}), damit der MST schließlich einheitlich vorliegt.
Anschließen kontrahiert PE $p_j$ den Graphen (\hyperref[kontrahiere-start]{Zeile} \ref{kontrahiere-start} - \ref{kontrahiere-end}), genau wie im sequenziellen Fall. Ein Beispiel zur Kontraktion eines Graphen ist in \cref{Boruvka-Img} zu sehen.
Um herauszufinden welche Knoten kontrahiert werden können, nutzen wir ein weiteres Array $P$ der Länge n. Hier schreibt $p_j$ in den i-ten Eintrag den Vorgängerknoten (\textbf{parent vertex}) des i-ten Knoten (\hyperref[parent-start]{Zeile} \ref{parent-start} - \ref{parent-end}).
Der Vorgänger ist hier der kleinste Knoten der über die leichtesten inzidenten Kanten erreichbar ist. Damit können in $O(n)$ alle Knoten mit dem selben Vorgänger kontrahiert werden.\\
Am Ende dieser Iteration kann optional $p_j$ noch die übrigen Parallelen Kanten zwischen zwei Knoten entfernt werden (\hyperref[removeParallel-code]{Zeile} \ref{removeParallel-code}). \\
Insgesamt wird das Vorgehen solange wiederholt, bis nur noch ein Graph mit genau einem Knoten übrig bleibt.\\
Anschließend muss der berechnete MST noch in die ursprüngliche Form zurück umgewandelt werden, da die MST Kanten durch das Vorgehen umbenannt wurden.
Wir haben diesen Schritt in $O(n)$ implementiert, indem unsere Kanten die Form $(s,t,w,s_{origin},t_{origin})$ hatten. Wir haben den originalen Start- und Endknoten zusätzlich gespeichert, sodass wir die Kanten wieder direkt in ihren Ausganszustand bringen können.

\begin{algorithm} 
\caption{\boruvkaStep(V, E, MST: Kantenliste)}
\begin{algorithmic}[1]
\label{BoruvkaStep}

\STATE E $\leftarrow$ localMST(E,V) //optional
\STATE $I_L$[|V|]
\STATE $I_L$ $\gets [\infty, ..., \infty]$ 
\FOR{\textbf{each} $e \in E$} \label{Incident-start}
    \IF{$e.w$ < $I_L$[$e.s$].$w$}
        \STATE $I_L$[$e.s$] $\gets$ $e$
    \ENDIF
    \IF{$e.w$ < $I_L$[$e.t$].$w$}
        \STATE $I_L$[$e.t$] $\gets$ $e$
    \ENDIF
\ENDFOR \label{Incident-end}
\STATE $E_I$ $\leftarrow$ AllreduceMinIncident(N, $I_L$) \label{Allreduce-code}
\IF{PE hat ID = 0} \label{addMST-start}
    \STATE MST $\gets$ MST $\cup$ $E_I$ 
\ENDIF \label{addMST-end}
\STATE P[|V|]
\FOR{$i\gets$ 0 bis |V|-1} \label{parent-start}
    \STATE P[i] $\gets$ $min_V$\{V ist über $E_I$ zu Knoten i verbunden\}
\ENDFOR \label{parent-end}
\STATE V $\gets$ relabelVertices($E_I$, P) \label{kontrahiere-start}
\STATE E $\gets$ relabelEdges($E_I$, P)\label{kontrahiere-end}
\STATE E $\gets$ removeParallelEdges(E) //optional \label{removeParallel-code}
\end{algorithmic}
\end{algorithm}



\begin{algorithm} 
\caption{\textsc{Bor{\r u}vka-Allreduce}(V, E): Kantenliste}
\begin{algorithmic}[1]
\label{Boruvka-Allreduce-Algo}

\STATE MST: Kantenliste
\WHILE{|V| > 1}
    \STATE Bor{\r u}vkaStep(V, E, MST)
\ENDWHILE

\RETURN getOriginEdges(MST) //Umbenennung der Kanten rückgängig machen
\end{algorithmic}
\end{algorithm}


\subsubsection{Eigenschaften und Komplexität}
\boruvkaAllreduce reduziert die Anzahl an Knoten in jeder Iteration um mindestens die Hälfte. Jeder Knoten wird anhand einer Kante mit einem oder mehreren Knoten zusammengefasst. Umso dichter der Graph also ist, desto wahrscheinlicher ist es dass mehr Knoten in einer Iteration zusammengefasst werden. Insgesamt ergibt sich damit, dass für \boruvkaAllreduce maximal $\log(n)$ Iterationen nötig sind.

Die Berechnung der leichtesten inzidenten Kanten ist in $O(m)$ möglich, der Allreduce benötigt $O(\alpha \log(p)+\beta \ell)$ im Alpha-Beta Modell. Wobei $\ell$ die Größe des Arrays mit den leichtesten inzidenten Kanten ist, also $\ell \in O(n)$.
Anhand der inzidenten Kanten kann man den Graph in $O(n + m)$ kontrahieren.


Sei $m'$ die maximale Anzahl an lokalen Kanten eines PEs.
Dann liegt der initiale \boruvkaStep \space in $O(m'+ \alpha \log(p)+\beta n)$.
Für jeden folgenden \boruvkaStep \space wird n mindestens um die Hälfte kleiner. Hierbei liegt die Summe aus $\sum_{i=i}^{\log n} \beta \frac{n}{2^i}$ in $O(\beta n)$.
Damit ergibt sich die gesamte Laufzeit von \boruvkaAllreduce in $O(\log n [m' + \alpha \log p]  +\beta n)$.




\subsection{Bor{\r u}vka-Mixed-Merge}
Jede Iteration von \boruvkaMixedMerge besteht aus einer abwechselnden Ausführung von einem \boruvkaStep \space gefolgt von einem \textsc{MergeStep}.
Damit verbindet dieser Algorithmus die Vorteile von \textsc{Merge-Local-MST} und \boruvkaAllreduce.
Der Aufwand von einem \boruvkaStep \space hängt von $m$ und $p$ ab und ein \textsc{MergeStep} ist auf möglichst dichten Graphen am effizientesten. Diese Eigenschaften werden durch \boruvkaMixedMerge zum Vorteil verwendet.
Denn durch das Vorgehen von \boruvkaMixedMerge, wird vor jedem \textsc{MergeStep} die Knotenmenge des Graphen mindestens halbiert und die Kantenanzahl sowie die aktiven PEs vor jedem (außer dem ersten) Brouvkastep um den Faktor $D$ reduziert.
Daher verspricht \boruvkaMixedMerge eine effiziente Ausführung in der Praxis.
Der Pseudocode von \cref{Boruvka-Mixed-Merge-Algo} zeigt die genaue Vorgehensweise von \boruvkaMixedMerge.

Insgesamt benötigt dieser Algorithmus $\lfloor \log_{D}($p$) \rfloor $ Iterationen, so wie \mergeMST, wegen dem Zusammenfügen der MST.
Zum Schluss müssen die MST Kanten wieder zu ihrer Ausgangsform umbenannt werden, da sich der Graph wie in \boruvkaAllreduce verändert hat.\\


\begin{algorithm} 
\caption{\textsc{Bor{\r u}vka-Mixed-Merge}(V, E, D: int): Kantenliste}
\begin{algorithmic}[1]
\label{Boruvka-Mixed-Merge-Algo}

\STATE MST: Kantenliste
\STATE lokalerMST $\gets$ localMST(E, V)
\FOR{i $\leftarrow$ 0 bis $\lfloor \log_{D}($p$) \rfloor $}
    \STATE Bor{\r u}vkaStep(V, lokalerMST, MST)
    \STATE MergeStep(V, MST, D)
\ENDFOR

\RETURN getOriginEdges(MST) //Umbenennung der Kanten rückgängig machen
\end{algorithmic}
\end{algorithm}

Die Laufzeit setzt sich insgesamt aus $\lfloor \log_{D}(p) \rfloor$  \textsc{Bor{\r u}vkaSteps} und einem \textsc{MergeSteps} zusammen. Sei $m \geq n$ und $m'$ die maximale Anzahl an lokalen Kanten eines PEs, dann liegt die gesamte Laufzeit in 
\begin{center}
$O(\alpha \log(p) + D\beta n + \log_D(p) \cdot D\alpha \cdot T_{seq}(n,Dn))$.
\end{center} 
Hierbei ist zu beachten, dass man den $\log(p/D^i)$ Term, der bei jedem \textsc{MergeStep} auftritt, asymptotisch mit $\log(p)$ abschätzen kann.


\subsection{Bor{\r u}vka-Then-Merge}
\boruvkaThenMerge führt mehrmals den \boruvkaStep\space durch und anschließend den \textsc{Merge-Local-MST} Algorithmus, um einen MST zu berechnen. Da die Kanten durch mehrere Iterationen von \boruvkaAllreduce angepasst wurden, müssen diese für die Ausgabe wieder in den Ausgangszustand gebracht werden. \\
Wie bei \boruvkaMixedMerge werden hier die Vorteile von \boruvkaAllreduce und \mergeMST hervorgehoben.
Da dieser Algorithmus damit anfängt mehrere \boruvkaStep \textsc{s} auszuführen, wird der Graph verkleinert ohne das bereits PEs inaktiv werden. Erst wenn nur noch höchstens $Border$ Knoten übrig sind, wird der restliche MST mittels \mergeMST berechnet. In diesem Fall hat jeder PE deutlich weniger Knoten, als in der Eingabe und ein \mergeStep\space kann \emph{schneller} abgearbeitet werden.

Wie oft man zu Beginn einen \boruvkaStep \space durchführt, hängt vom Anwendungsfall ab. Zu viele Iterationen von \boruvkaAllreduce führen zu keiner Verbesserung, da sich der Algorithmus im kaum von \boruvkaAllreduce unterscheidet. Zu wenige Iterationen führen dazu, dass sich \boruvkaThenMerge sehr ähnlich zu \mergeMST verhält.\\
Dehne et al. \cite{dehne1998practical} führen \boruvka solange durch bis weniger als $n/\log_{D}^2(p)$ Knoten übrig sind und daher haben wir das auch in unserer Implementierung übernommen.



\begin{algorithm} 
\caption{\textsc{Bor{\r u}vka-Then-Merge}(V, E, D: int, Border: int): Kantenliste}
\begin{algorithmic}[1]
\label{Boruvka-Then-Merge-Algo}

\STATE MST: Kantenliste
\STATE lokalerMST $\gets$ localMST(E, V)
\WHILE{|V| $\geq$ Border}
    \STATE Bor{\r u}vkaStep(V, lokalerMST, MST)
\ENDWHILE

\STATE MST $\gets$ Merge-Local-MST(V, MST, D)

\RETURN getOriginEdges(MST) //Umbenennung der Kanten rückgängig machen
\end{algorithmic}
\end{algorithm}





Die gesamte Laufzeit ergibt sich durch $B$ mal einen \boruvkaStep \space und $\lfloor \log_{D}($p$) \rfloor$ mal einen \textsc{MergeStep}. 
Sei $m'$ die maximale Anzahl an lokalen Kanten eines PEs, $T_{seq}$ die Laufzeit für eine lokale MST Berechnung und es gelte $m \geq n$.
Dann liegt \boruvkaThenMerge in 
\begin{center}
$O(B \cdot (m' + \alpha \log p) + \beta n
+ \log_{D}(p)  (T_{seq}(\frac{n}{2^B},\frac{n}{2^B}) + D(\alpha + \beta \frac{n}{2^B}))
+ T_{seq}(\frac{n}{2^B},m')
)$
\end{center}

\section{Implementierung}\label{Implementierung}
Auch wenn man mittels asymptotischer Laufzeitanalyse Algorithmen sinnvoll miteinander vergleichen kann, geben sie nur einen Einblick in das theoretische Verhalten. Insbesondere stimmen die theoretischen Modelle nicht mit den real-existierenden Maschinen überein. Damit sind diese oft sehr pessimistisch und die tatsächlichen Ergebnisse können in der Praxis stark schwanken.\\
Aus diesem Grund klären wir im Folgenden Details, die unsere Implementierung betroffen haben sowie die Art und Weise wie wir unsere Eingabegraphen konstruiert haben.\\
Die Programmierung unserer Algorithmen \footnote{\url{https://github.com/u-Texon/parallel-dense-mst}} ist in der Programmiersprache $C$++ einstanden unter Verwendung verschiedener Bibliotheken. So haben wir für die Kommunikation zwischen Prozessoren das \textbf{M}essage \textbf{P}assing \textbf{I}nterface (MPI) verwendet. MPI ermöglicht einen einfachen und gezielten Umgang und bietet auch eine Vielzahl an kollektiven Operationen an.
Für die Generierung der Eingabegraphen haben wir \textit{KaGen}\footnote{\url{https://github.com/KarlsruheGraphGeneration/KaGen}} \cite{funke2017communication,HubSan2020RMAT} verwendet und zum Sortieren (z.B bei Kruskals Algorithmus) 
\textit{IpsoSort}\footnote{\url{https://github.com/SaschaWitt/ips4o}} \cite{axtmann2017}. Dieser Sortieralgorithmus kann sowohl parallel als auch sequenziell verwendet werden. Da wir in den Algorithmen nur sequenziell Sortieren, haben wir den parallelen Algorithmus auch nicht verwendet.


\subsection{Zeitmessung}
Damit die Laufzeitmessung unserer Algorithmen auf mehreren Prozessoren funktioniert, wird vor dem Start-Aufruf des Timers immer eine Barriere aufgerufen, die so lange das Programm anhält bis alle Prozessoren an dieser Stelle im Code angekommen sind (und auch die Barriere aufrufen). Erst ab diesem Punkt fängt die Zeit an zu laufen. Mit demselben Vorgehen wird vor dem Stoppen des Timers auch über eine Barriere sichergestellt, dass alle Prozessoren am Ende der Ausführung angekommen sind. Erst dann wird die Zeit angehalten.

Das bedeutet im Wesentlichen, dass ein Timer Aufruf auch immer zwangsläufig ein Barrieren Aufruf beinhaltet. Somit wird zusätzliche Zeit benötigt, um darauf zu warten, dass alle Prozessoren dieselben Codebereiche ausgeführt haben. \\
Da wir für alle Algorithmen nicht nur die gesamte Laufzeit, sondern auch einzelne Phasen der Algorithmen messen wollen, sind wir wie folgt vorgegangen: Für die Evaluation jeglicher Weakscaling Ergebnisse, haben wir nur die Gesamtlaufzeit der Algorithmen gemessen. Das bedeutet wir haben nur einen Timeraufruf vor und nach dem Algorithmus ausgeführt, damit zusätzliche Synchronisatione des Timers keine Auswirkung auf die Algorithmen haben.\\
Bei den Messungen der einzelnen Phasen eines Algorithmus, können wir dieses Problem allerdings nicht umgehen. Daher sollte man beachten, dass einerseits die Gesamtlaufzeit ggf. höher ist als bei den Weakscaling Ergebnissen. Anderseits können Phasen wie das Allreduce bei dem \boruvkaAllreduce Algorithmus geringer erscheinen als sie im Normalfall sind. Das liegt daran, dass für ein Allreduce eine Synchronisation aller Prozessoren für den Nachrichtenaustausch stattfindet. Sollte also ein einzelner Prozessor länger für eine vorherige Aufgabe benötigen als die übrigen, so gehört zum Allreduce das Warten auf diesem Prozessor dazu. Da wir in unserem Fall aber über den Timer Aufruf alle Prozessoren vor dem Allreduce Aufruf synchronisieren, dauert das Warten auf einen einzelnen Prozessor im Anschluss nicht so lange wir üblich.


\subsection{Generierung der Graphen}
Die von uns verwendete KaGen Bibliothek generiert abhängig von der Anzahl an $n$ Knoten, $m$ Kanten, $p$ Prozessoren und Graphtyp $t$ einen bestimmten Graphen. Für diese Generierung muss $p,m,n \in \{2^k | k \in \mathbb{N}\}$ gelten. Daher haben wir unsere Implementierung nur mit einer Zweierpotenz an Prozessoren, Kanten und Knoten evaluiert und getestet. Auch wenn die Algorithmen mit beliebigen Eingaben und Konfigurationen funktionieren. \\
Nach der Generierung liegt der Graph global lexikographisch sortiert auf den einzelnen Prozessoren vor. Zusätzlich wird zu jeder Kante $(s,t,w)$ auch die Rückkante $(t,s,w)$ generiert. 
Allerdings können die verwendeten Algorithmen auch auf zufällig permutierten Kantenlisten arbeiten. Weil wir eventuelle Auswirkungen einer sortierten Eingabe ausschließen möchten, wurde die global sortierte Kantenliste zufällig auf alle PEs verteilt. 
Zu den wesentlichen Graph Typen, die wir verwendet haben gehören GNM, RHG und Pair. 
Bei der Generierung von GNM Graphen mit $n$ Knoten wird $m$ mal zwischen zwei zufälligen Knoten eine Kante gezogen.
Bei RHG Graphen hingegen werden zunächst $n$ Punkte auf einem Kreis mit Radius $R$ gesetzt. Mit weiteren Parametern kann gesteuert werden, wie nah diese Punkte an dem Kreisradius liegen sollen. Zwischen zwei Knoten verläuft eine Kante, wenn der \emph{hyperbolische Abstand} kleiner als $R$ ist. \\

\subsubsection{Pair Graph}
Der Pair Graph ist ein von uns erstellter Graph beruhend auf der Idee von Dehne und Götz \cite{dehne1998practical} mit dem Ziel möglichst viele Iterationen in \boruvkasAlgorithmus zu erzwingen.\\
In jeder Iteration von \boruvkasAlgorithmus verringert sich die Knotenanzahl um mindestens die Hälfte. Besonders auf dichten Graphen, können in einem Schritt eine Vielzahl von Knoten kontrahiert werden, sodass \boruvkasAlgorithmus deutlich weniger als $\log(n)$ Iterationen benötigt. Um aber explizit dieses Szenario betrachten zu können, haben wir mit der Beschreibung  einen eigenen Pair Graphen generiert.

Das Vorgehen zum Generieren eines solchen Graphen ist relativ simpel, besonders für uns, da wir nur Graphen mit $n = 2^k$ betrachten: \\
Wir erstellen zunächst zwischen jedem Zweierpaar an Knoten eine Kante mit Gewicht 1, anschließend zwischen jedem Viererpaaren mit Gewicht 2, dann zwischen Achterpaaren mit Gewicht 3 und so weiter. Somit generieren wir insgesamt n-1 Kanten und können weitere zufällige Kanten (mit minimalem Gewicht von n) hinzufügen. 
Eine schematische Darstellung eines Pair Graphen mit 8 Knoten ist in Abbildung \cref{Pair-Graph-Img} dargestellt.

Durch die geringen Kantengewichte stellen wir sicher, dass es sich bei all diese Kanten genau um die MST-Kanten des Graphen handelt. Im ersten Borvkaschritt werden also alle Knotenpaare die über eine Kante mit Gewicht eins verbunden sind kontrahiert, im nächsten Schritt diejenigen Paare, die über Gewicht zwei verbunden sind und so weiter. Damit ergibt sich für einen Pair Graphen immer die maximal mögliche Anzahl von $\log(n)$ Iterationen für Bor{\r u}vkas Algorithmus.

\begin{figure}[H]
    \centering
    \includesvg[width=12cm]{Figures/PAIR-Graph.svg}
    \caption{Pair Graph mit $n=8$}
    \label{Pair-Graph-Img}
\end{figure}



\section{Ergebnisse und Evaluierung}\label{Evaluierung}
Die Algorithmen aus \hyperref[Algorithms]{Kapitel} \ref{Algorithms} haben wir auf dem SuperMUC-NG Supercomputer ausgeführt. Dieser Supercomputer ist ein Cluster bestehend aus 6336 \emph{Compute-Nodes}. Ein Compute-Node besteht aus 48 Prozessoren und besitzt 96 GByte Hauptspeicher. Untereinander sind die Compute-Nodes über das OmniPath-Network mit einer Bandbreite von 100GBit/s verbunden. Als Compiler haben wir GCC 11.2 und die MPI Version 4.0.7 verwendet.\\
 Die folgenden Angaben zur Kantenmenge bezieht sich immer auf den \textbf{ungerichteten} Graphen. Allerdings ist die Kantenmenge für die Ausführung genau doppelt so Groß, da wir die jeweiligen Rückkanten mit generiert haben. Insgesamt haben wir jeweils 5 durchläufe von jedem Algorithmus ausgeführt und den jeweils ersten verworfen, um \emph{Warmup-Effekte} auszuschließen. Aus den restlichen vier Laufzeiten haben wir schließlich den Durchschnitt berechnet.

\subsection{Lokalen MST Berechnung}
Weil lokale Berechnung von MSTs einen wesentlichen Teil von den verteilt parallelen Algorithmen ausmachen, muss dieser so effizient wie möglich sein. 
Daher benutzen wir für die lokale MST Berechnung 
\textsc{Filter-Kruskal} als Alternative zum klassischen Kruskal. Dieser ermöglicht in der Praxis eine deutlich bessere Laufzeit auf einer breiten Masse an Eingaben. 
In \cref{Filter-Kruskal-Img} ist ein direkter Vergleich zwischen Kruskals Algorithmus und \textsc{Filter-Kruskal} zu sehen. Hierbei ist auf der X-Achse der Knotengrad ($m/n$) und auf der Y-Achse die benötigte Zeit pro Kante ($m$/Laufzeit) abgebildet. Die Ausführung hat auf einem (GNM) Graph mit $2^{18}$ Knoten und $2^{18}$ bis $2^{24}$ Kanten stattgefunden.
Es ist zu eindeutig zu erkennen, dass \textsc{Filter-Kruskal} effizienter ist als Kruskals Algorithmus. Dass liegt daran, dass \textsc{Filter-Kruskal} im Gegensatz zu Kruskals Algorithmus nicht alle Kanten sortiert werden.
Aufgrund der besseren Laufzeit verwenden wir für die folgenden Experimente immer \textsc{Filter-Kruskal} als Basis Algorithmus für die Bestimmung von lokalen MSTs.

\begin{figure}[H]
    \centering
    \includesvg[width=10cm]{ergebnisse/sequential/sequential.svg}
    \caption{Vergleich zwischen Kruskals Algorithmus und \textsc{Filter-Kruskal}}
    \label{Filter-Kruskal-Img}
\end{figure}



\subsection{Sehr Dichte Graphen}\label{dense}
Die behandelten MST Algorithmen sind für sehr dichten Graphen ausgelegt. Einerseits muss die Knotenmenge auf jedem PE repliziert werden können und andererseits sind jegliche Iterationen von \boruvkaAllreduce und \mergeMST auf dichten Graphen besonders effektiv. Insbesondere werden auf dichten Graphen in einem \boruvkaStep mehr Knoten kontrahiert und bei einem \mergeStep mehr Kanten durch lokale MST Berechnungen entfernt.\\
In \cref{WeakSkale-dense-Img} ist die Laufzeiten von \textsc{Bor{\r u}vka-Allreduce}, \textsc{Merge-Local-MST}, \boruvkaMixedMerge und \textsc{Bor{\r u}vka-Then-Merge} abgebildet. Hierbei handelt es sich um \enquote{weak scaling}, d.h die Anzahl an Kanten im Graphen skaliert mit der Anzahl an PEs. Eine optimale Effizienz der Algorithmen würde sich also als eine konstante Linie widerspiegeln. In diesem Fall haben wir die Algorithmen auf einem RHG und einem GNM Graph mit $2^{18}$ Knoten und $2^{22}$ Kanten pro PE ausgeführt. Auf der X-Achse ist die Anzahl an PEs von 1 bis 2048 angegeben und auf der Y-Achse sehen wir die durchschnittliche Laufzeit in Millisekunden. \boruvkaAllreduce berechnet nur zu Beginn einen lokalen MST und entfernt keine parallelen Kanten. Außerdem ist der verwendete Treefactor für alle Algorithmen $D=2$.

\begin{figure}[H]
    \centering
    \includesvg[width=8.33cm]{ergebnisse/dense/weak-scale-RHG.svg}
    \includesvg[width=7.67cm]{ergebnisse/dense/weak-scale-GNM.svg}
    \caption{Weakscaling auf dichten Graphen}
    \label{WeakSkale-dense-Img}
\end{figure}

Wir erkennen, dass auf dem GNM Graphen Merge-Local-MST mit ca. 0,75 Sekunden bei 2048 PEs am schlechtesten abschneidet, wobei die übrigen Algorithmen alle mit ca. 0,5 Sekunden fast identisch sind. Auf dem RHG Graphen ist das Verhalten ebenso zu erkennen.
Es ist erwartbar, dass Merge-Local-MST am langsamsten ist. Immerhin werden nur in der ersten Iteration alle PEs gänzlich genutzt. Es überrascht allerdings, dass es kaum einen Unterschied zwischen den restlichen Algorithmen gibt. Hierfür betrachten wir in \cref{Explicit-dense-Img} explizit die Laufzeiten der einzelnen Algorithmen. Diese sind in die jeweiligen Iterationen und Phasen aufgeteilt.



\begin{figure}[H]
    \centering
    \includesvg[width=8.11cm]{ergebnisse/dense/merge_bar.svg}
    \includesvg[width=7.89cm]{ergebnisse/dense/boruvkaMerge_bar.svg}

    \includesvg[width=7.89cm]{ergebnisse/dense/boruvka_bar.svg}
   \includesvg[width=8.11cm]{ergebnisse/dense/mixedMerge_bar.svg}
    \caption{Ansicht für einzelne Algorithmen}
    \label{Explicit-dense-Img}
\end{figure}


Hier ist eindeutig, dass die Laufzeit fast ausschließlich von lokalen MST Berechnung stammt. Bei \boruvkaAllreduce dominiert einerseits die Laufzeit der ersten Iteration und andererseits haben folgenden Iterationen kaum eine Auswirkung auf die Laufzeit.
So benötigt der lokale MST mit ca. 0.38 Sekunden etwa 80\% der gesamten Laufzeit. Da sowohl \boruvkaAllreduce als auch \boruvkaThenMerge und \boruvkaMixedMerge mit einem \boruvkaStep beginnen und dieser Schritt fast die gesamte Laufzeit ausmacht, ist es logisch, dass diese Algorithmen fast gleich schnell sind.


Wenn aber nur die lokale MST Berechnung relevant für den globalen MST ist, warum ist dann nicht auch \mergeMST gleich schnell? Immerhin besteht dieser Algorithmus fast nur aus diesen Schritten. \\
Um diese Frage zu beantworten, betrachten wir die Boxplots in \cref{Boxplot-dense-Img}.
Diese Graphik zeigt die Anzahl an Knoten und Kanten für jeden PE nach den jeweiligen Iterationen.



\begin{figure}[H]
    \centering
    \includesvg[width=8.45cm]{ergebnisse/dense/boruvka_box.svg}
    \includesvg[width=7.55cm]{ergebnisse/dense/merge_box.svg}
    \caption{Boxplots von \mergeMST und \boruvkaAllreduce}
    \label{Boxplot-dense-Img}
\end{figure}

Da der Graph besonders dicht war, konnte \mergeMST in der ersten Iteration besonders viele Kanten bereits aussortieren. Allerdings ist das bei den folgenden Iterationen nicht mehr möglich. Nach der ersten Iteration hält jeder PE nur noch ca. $n=2^{18}$ Kanten. In den nächsten Iterationen werden also nur zwei MSTs mit $2^{18}$ Kanten verschmolzen bis nur noch ein PE aktiv ist.
Während also die anderen Algorithmen mit einem \boruvkaStep den Graph schrumpfen und die restlichen Kanten aussortieren, benötigt Merge-Local-MST viel Zeit mit weiteren lokalen Berechnungen, die kaum zielführend sind. 




\subsection{Verschmelzen von lokalen MSTs}\label{eval-treefactor}
Der Treefactor (oder $D$) gibt an wie viele lokale MSTs in einer Iteration von Merge-Local-MST verschmolzen werden. Ein kleinerer Treefactor sorgt somit für mehr Iterationen, aber dafür sind weniger Prozessoren im Anschluss inaktiv.\\
In \cref{Treefactor-Img} sehen wir \mergeMST mit fünf verschiedenen werten für $D$.
Dabei unterscheiden wir für die Treefactor Werte zwischen 2, 3, 4, 8 und 16 auf einem GNM Graphen mit $2^{18}$ Knoten und $2^{20}$ Kanten pro PE unter Verwendung von \textsc{Filter-Kruskal}.


\begin{figure}[H]
    \centering
    \includesvg[width=8cm]{ergebnisse/treeFactor/mergeMST-RHG.svg}
    \includesvg[width=8cm]{ergebnisse/treeFactor/mergeMST-GNM.svg}
    
    \caption{Laufzeit von Merge-Local-MST mit verschiedenen Treefactor Werten}
    \label{Treefactor-Img}
\end{figure}


Dehne und Götz \cite{dehne1998practical} haben in ihren Experimenten mit D=3 die besten Ergebnisse erzielt. 
Wie man in \cref{Treefactor-Img} aber sieht, war das bei uns nicht der Fall. Hier hat Merge-Local-MST mit $D=4$ bzw. $D=2$ die beste Laufzeit. 
Die Experimente haben gezeigt, dass sich bereits ein größerer Treefactor als 4 negativ auf die Laufzeit auswirkt. Dies wird durch diese Graphik auch nochmal bestätigt. Bereits mit D=8 ist die Laufzeit ca. 10\% und bei D=16 schon ca.50\% höher als bei D=2.\\
Für die meisten Eingaben war $D=2$ am effizientesten, weshalb wir diesen Wert die restlichen Experimente verwendet haben.



\subsection{Ergebnisse auf unterschiedlichen Graphen}
Wir möchten nun beobachten, wie sich die Algorithmen auf den unterschiedlichen Graphen verhalten.
Dafür betrachten wir die von uns implementierten Algorithmen auf drei unterschiedlichen Graphtypen. Wir nutzen als Eingabegraphen GNM, RHG und Pair mit $2^{18}$ Knoten und $2^{17}$ bis $2^{20}$ Kanten pro PE.


\subsubsection{GNM}
\cref{GNM-graphs-Img} zeigt die Ergebnisse unserer Experimente mit dem GNM Graphen als Eingabe. \\
Die gute Skalierung der Algorithmen wird dadurch deutlich, dass, obwohl die Eingaben bis zu vier Mal mehr Kanten besitzen, die Laufzeiten dennoch in der selben Größenordnung liegen. So ist \textsc{Merge-Local-MST} bei $2^{20}$ Kanten pro PE nur 10\% langsamer als bei $2^{20}$ Kanten auf 2048 Prozessoren. Weiterhin zeigen die Experimente zu unserer Erwartung, dass \textsc{Merge-Local-MST} die größte Laufzeit von allen Algorithmen benötigt. Das liegt daran, \textsc{Merge-Local-MST} nach der ersten Iteration nur noch deutlich langsamer Fortschritte macht, wie es auch z.B in \hyperref[dense]{Kapitel} \ref{dense} zu beobachten ist.\\
Im Gegensatz zu \hyperref[dense]{Kapitel} \ref{dense} ist der Graph deutlich weniger dicht
und die lokale MST Berechnung domioniert die Laufzeit weniger stark.
Nun sind Unterschiede zwischen \boruvkaAllreduce, \boruvkaThenMerge und \boruvkaMixedMerge besser erkennbar. So sieht man bei $2^{20}$ Kanten pro PE, dass \boruvkaAllreduce über 10\% langsamer ist, als \boruvkaThenMerge und \textsc{Bor{\r u}vka-Mixed-Merge}. Dennoch liegen die Laufzeiten dieser drei Algorithmen nah bei einander, da nach wie vor wenige Iterationen von \boruvkaAllreduce einen großen Teil der Laufzeit ausmachen.
\newpage

\begin{figure}[H]
    \centering
    \includesvg[width=8cm]{ergebnisse/graphs/gnm-17.svg}
    \includesvg[width=8cm]{ergebnisse/graphs/gnm-18.svg}

    \includesvg[width=8cm]{ergebnisse/graphs/gnm-19.svg}
    \includesvg[width=8cm]{ergebnisse/graphs/gnm-20.svg}
    \caption{GNM Graphen mit $2^{18}$ Knoten und $2^{17}$ bis $2^{20}$ Kanten pro PE}
    \label{GNM-graphs-Img}
\end{figure}


\subsubsection{RHG}
In \cref{RHG-graphs-Img} sieht man die Laufzeiten der Algorithmen mit den selben Konfigurationen, aber diesmal auf dem RHG Graph. Wie bei den GNM Graphen bleibt die gute Skalierung erhalten und auch der Verlauf der Laufzeiten sieht identisch aus. Hierbei liegen die Laufzeiten von \boruvkaAllreduceNoSpace, \boruvkaThenMerge und \boruvkaMixedMerge noch näher bei einander als auf GNM Graphen. Vermutlich sorgt die Struktur des Graphen für eine erhöhte Effizienz bei Iterationen von \boruvkaAllreduceNoSpace. Damit sind die Aufwände der drei Algorithmen größtenteils nur von \boruvkaAllreduce abhängig. \\
Die Unterschiede zwischen den drei Algorithmen werden erst deutlich, wenn mehr Iterationen von \boruvkaAllreduce nötig sind, um den MST zu berechnen. Um diesen Fall zu betrachten bietet sich der Pair Graph an.

\newpage


\begin{figure}[H]
    \centering
    \includesvg[width=8cm]{ergebnisse/graphs/rhg-17.svg}
    \includesvg[width=8cm]{ergebnisse/graphs/rhg-18.svg}
    \includesvg[width=8cm]{ergebnisse/graphs/rhg-19.svg}
    \includesvg[width=8cm]{ergebnisse/graphs/rhg-20.svg}
    \caption{RHG Graphen mit $2^{18}$ Knoten und $2^{17}$ bis $2^{20}$ Kanten pro PE}
    \label{RHG-graphs-Img}
\end{figure}





\subsubsection{Pair}
Schließlich betrachten wir die Laufzeiten der Algorithmen auf dem Pair Graphen in \cref{Pair-graphs-Img}. Im Gegensatz zu GNM und RHG Graphen, sind hier deutliche Unterschiede in der Effizienz der Algorithmen zu erkennen.\\
\boruvkaAllreduce ist nun der langsamste Algorithmus. Das ist in diesem Fall sinnvoll, da der Pair Graph den schlechtesten Eingabegraphen für \boruvkasAlgorithmus darstellt, indem eine maximale Anzahl an Iterationen erzwungen wird.
Der effizienteste Algorithmus ist hingegen \boruvkaMixedMergeNoSpace. Durch die Abwecheslnde Ausführung von einem \boruvkaStep und einem \mergeStep wird die Kantenanzahl vor jedem \boruvkaStep verringert, sodass dieser deutlich weniger Zeit benötigt.\\
\boruvkaThenMerge ist hier etwas langsamer, da die ersten Iterationen von \boruvkasAlgorithmus immer noch starke Auswirkungen haben. Allerdings lohnen sie sich dennoch insofern, dass \boruvkaThenMerge noch effizienter als \mergeMST ist.

\newpage

\begin{figure}[H]
    \centering
    \includesvg[width=8cm]{ergebnisse/graphs/pair-17.svg}
    \includesvg[width=8cm]{ergebnisse/graphs/pair-18.svg}
    \includesvg[width=8cm]{ergebnisse/graphs/pair-19.svg}
    \includesvg[width=8cm]{ergebnisse/graphs/pair-20.svg}
    \caption{Pair Graphen mit $2^{18}$ Knoten und $2^{17}$ bis $2^{20}$ Kanten pro PE}
    \label{Pair-graphs-Img}
\end{figure}


\subsection{Entfernen von parallelen Kanten}\label{remove-Section}
Nachdem wir in einem \boruvkaStep den Graph kontrahiert und die Kanten umbenannt haben, bleiben viele Kanten übrig die man verwerfen kann. Einerseits gibt es Kanten der Form $(s,t,w)$, bei denen die Knoten $s$ und $t$ in der gleichen Komponente lagen und diese im Anschluss die Form $(s',s',w)$ haben. Diese Schlingen sind offensichtlich niemals Teil eines MSTs und werden in unserer Implementierung bei der Umbenennung direkt entfernt.\\
Andererseits können auch viele \emph{parallele} Kante auftreten, also mehrere Kanten die die gleichen Knoten Verbinden wie z.B die Kante $(0,1,7)$ und $(0,1,12)$. Hierbei muss nur diejenige Kante beibehalten werden, welche das geringste Gewicht besitzt. Da dieser Schritt allerdings nur Kanten entfernt, die niemals zum MST hinzugefügt werden, ist dieses Vorgehen bei einem \boruvkaStep optional. \\
Nun fragen wir uns, ob bzw. wie sehr sich dieser Schritt lohnt und auf welche Weise man diesen am effizientesten implementieren kann.\\
Für unsere Implementierung zum Entfernen der parallelen Kanten haben wir zunächst alle Kanten nach Startknoten, Endknoten und dann nach Gewicht sortiert. Anschließend iterieren wir ein weiteres mal über die Kanten und behalten nur das erste Vorkommen einer Kante. Damit benötigt removeParallelEdges $O(m\log m)$ Operationen zum Sortieren plus $O(m)$ Operation für das Entfernen.\\
Alternativ dazu können wir auch zu Beginn von jedem \boruvkaStep mittels Filter-Kruskal einen lokalen MST berechnen.\\


\begin{figure}[H]
    \centering
    \includesvg[width=5.3cm]{ergebnisse/remove/boruvka-default.svg}
    \includesvg[width=5.3cm]{ergebnisse/remove/boruvka-remove.svg}
    \includesvg[width=5.3cm]{ergebnisse/remove/boruvka-mst.svg}
    \caption{\boruvkaAllreduce mit und ohne optionaler Kantenreduktionen}
    \label{RemoveParallel-Img}
\end{figure}

Wir haben in \cref{RemoveParallel-Img} die Laufzeiten von \boruvkaAllreduce auf einem GNM Graphen mit $2^{19}$ Knoten und $2^{19}$ Kanten pro PE verglichen. Der linke Graph zeigt die Laufzeit von \boruvkaAllreduce ohne das Entfernen von parallelen Kanten und in der Mitte mit dem Entfernen. Rechts sieht man die Alternative mittels \textsc{Filter-Kruskal}. Beachte, dass wir bei allen drei Durchläufen immer zu Beginn einen lokalen MST berechnen, da dieses Vorgehen generell sehr effizient ist.\\
Wie man sieht, bildet das Entfernen der Parallelen kanten einen großen Nachteil im Vergleich zu den andern beiden Alternativen. Scheinbar werden verhältnismäßig wenige Kanten aussortiert und das benötigte Sortieren aller umbenannten Kanten hat deutliche negative Auswirkungen auf die Laufzeit.\\
Weiterhin fällt auf, dass \textsc{Filter-Kruskal} hingegen sehr effizient ist. Bei genauerer Überlegung wird klar, dass \textsc{Filter-Kruskal} mit einer erwarteten Laufzeit von $O(m)$ nicht nur schneller als unsere Implementierung in $O(m \log m)$ ist, sondern potenziell noch mehr Kanten als nur die parallelen entfernt. \\
Damit hat sich herausgestellt, dass sich Filter-Kruskal in der Praxis als eine sehr gute Methode entpuppt um parallele Kanten zu entfernen.\\
Um unsere Implementierung in der Praxis zu verbessern ist es weiterhin möglich Hash Tabellen zu verwenden. Dafür haben wir einen Teil (ca. 5\%) der leichtesten Kanten in einer Hash Tabelle gespeichert und anschließend für die übrigen Kanten überprüft, ob diese bereits in der Tabelle vorkommen. Wenn sie bereits enthalten sind, kann man sie entfernen, da sie ein höheres Gewicht haben. Im Anschluss haben von allen übrigen Kanten mit unserer $O(m \log m)$ Implementierung die parallelen Kanten entfernt. \\
In unseren Experimenten hat sich dieses Vorgehen nicht als vorteilhaft herausgestellt, weswegen wir dieses wieder verworfen haben.





\subsection{Nachrichten Überlagern}
Eine Möglichkeit, um einen \boruvkaStep zu beschleunigen, ist es den Allreduce Aufruf mit der lokalen MST Berechnung zu überlagern. Da sowohl Synchronisation und Nachrichtenübertragung sowie das lokale Berechnen von MSTs einen Großteil von einem \boruvkaStep ausmachen, möchten wir dies zu unserem Vorteil nutzen.\\
Unsere Idee war es zunächst mittels \emph{Hyper-Threading} zwei Threads pro PE zu starten, sodass ein Thread den Allreduce Aufruf durchführt während der andere den lokalen MST berechnet. Bei Hyper-Threading laufen mehrere Threads auf einem Prozessor und können den Prozessor effizienter ausnutzen, indem z.B ein Thread auf dem Prozessor läuft während der andere auf eine Antwort oder Ähnliches warten muss.
Eine naive Umsetzung von diesem Vorgehen konnte in der Praxis keinen Vorteil erzielen, da unsere Implementierung nicht ohne weiteres auf Hyper-Threading ausgelegt war. Außerdem benötigt dieses Vorgehen, dass in jedem \boruvkaStep die leichtesten inzidenten Kanten berechnet werden \textbf{bevor} die lokale MST Berechnung durchgeführt wird. Damit benötigt die Berechnung der leichtesten inzidenten Kanten mehr Zeit als gewöhnlich, da über mehr Kanten iteriert werden muss.
Weil die Zeit für diese Abschlussarbeit knapp wurde, konnten wir das Vorgehen eines \boruvkaStep mit Hyperthreading nicht genauer  analysieren und die Implementierung darauf anpassen. Um dennoch untersuchen zu können, ob das Überlagern von Nachrichten in der Praxis effizient ist, haben wir stattdessen zusätzliche PEs für den Allreduce Aufruf angefordert. Hierbei haben wir einem MPI Aufruf zwei PEs zugeordnet, damit in jedem \boruvkaStep ein PE den lokalen MST berechnet, während der andere den Allreduce Aufruf durchführt. Insofern wartet die Hälfte der PEs außerhalb der Nachrichtenüberlagerung.
Insgesamt haben wir die Experimente mit der doppelten Anzahl an PEs ausgeführt.

Der Ansatz der Nachrichtenüberlagerung lohnt sich am meisten, wenn der Allreduce Aufruf und die lokale MST Berechnung einen großen Bestandteil der Laufzeit ausmachen.
Es hat sich in \cref{remove-Section} ergeben, dass es sich lohnt in jedem \boruvkaStep einen lokalen MST mit \textsc{Filter-Kruskal} zu berechnen, anstatt parallele Kanten zu entfernen. Das bedeutet im Wesentlichen, dass es einen Vorteil verschafft in jedem \boruvkaStep einen lokalen MST zu berechnen, so wie es bei der Nachrichtenüberlagerung der Fall ist. Ansonsten würde das Überlagern von Nachrichten einen Nachteil bilden.\\
Zusätzlich muss der lokale MST einen ähnlich hohen Aufwand bilden wie der Allreduce Aufruf. Das ist aber nicht immer der Fall. Unsere Experimente haben ergeben, dass für $m/p \approx n$ der Allreduce Aufruf ungefähr so lange wie \textsc{Filter-Kruskal} benötigt. Um dies zu verdeutlichen, zeigt \cref{X-Img} die Laufzeit von drei \boruvkaAllreduce Durchläufen mit jeweils 1000 PEs. Die Ausführung hat auf einen GNM Graphen mit $2^{18}$ Knoten und $2^{17}$, $2^{18}$ und $2^{19}$ Kanten pro PE stattgefunden.


\begin{figure}[H]
    \centering
    \includesvg[width=5.3cm]{ergebnisse/overlap/boruvka-17.svg}
    \includesvg[width=5.3cm]{ergebnisse/overlap/boruvka-18.svg}
    \includesvg[width=5.3cm]{ergebnisse/overlap/boruvka-19.svg}
    \caption{Vergleich zwischen $m/p= 2^{17},  2^{18}$ und $2^{19}$ mit $n=2^{18}$}
    \label{X-Img}
\end{figure}

Hierbei erkennt man, dass bei $m/p=2^{17}$ und $m/p=2^{18}$ die lokale MST Berechnung ähnlich viel Zeit wie der Allreduce Aufruf gebraucht hat. Daher betrachten wir auf diesen beiden Eingaben in \cref{Overlap-Img} die Laufzeiten \boruvkaAllreduce mit und ohne Nachrichtenüberlagerung. Wir führen \boruvkaAllreduce ohne Nachrichtenüberlagerung auch auf doppelt so vielen PE's aus, nutzen aber die extra PEs nicht. 
Damit sorgen wir für gleiche Bedingungen, da in diesem Fall ggf. mehr Speicher für die einzelnen PEs zur Verfügung steht. Dadurch sind die Laufzeiten in \cref{Overlap-Img} außerdem noch ein Stück schneller als in \cref{X-Img}.

\begin{figure}[H]
    \centering
    \includesvg[width=8cm]{ergebnisse/overlap/overlap-17.svg}
     \includesvg[width=8cm]{ergebnisse/overlap/overlap-18.svg}
    \caption{Vergleich von \boruvkaAllreduce mit und ohne Nachrichtenüberlagerung}
    \label{Overlap-Img}
\end{figure}

Man erkennt, dass auf diesen beiden Eingaben die Überlagerung von Nachrichten einen Vorteil gegenüber zu einem üblichen \boruvkaAllreduce hat. Dabei ist eine Beschleunigung von biszu 30\% auf 1024 Prozessoren zu beobachten. Auch wenn die Bedingung and die Eingabe strikt ist, ist dieser Ansatz dennoch vielversprechend und effizient.\\ 


\newpage
\
\newpage
\section{Fazit}\label{Fazit}

Die Experimente haben gezeigt, dass die von Dehne et al. \cite{dehne1998practical} entwickelten Algorithmen in der Praxis auch auf mehreren tausend Prozessoren effizient skalieren. Während \mergeMST selten kürzere Laufzeiten als \boruvkaAllreduce besitzt, so waren \boruvkaMixedMerge und \boruvkaThenMerge auf vielen verschiedenen Eingaben die effizientesten Algorithmen. 
Besonder \boruvkaMixedMerge hat auf dem Pair Graphen, der viele Iterationen von \boruvkaAllreduce benötigt, die höchste Effizienz gezeigt. \\
Zusätzlich konnten wir einen deutlichen Einfluss auf die verteilt parallelen Algorithmen durch lokalen MST Berechnungen beobachten. Nicht nur bildet ein effizienter sequenzieller MST Algorithmus eine wichtige Basis für verteilte MST Algorithmen. Sondern machen diese auch bei sehr dichten Graphen einen Großteil der Laufzeit aus. Dabei war die Verwendung von \textsc{Filter-Kruskal} gegen über Kruskals Algorithmus ein wesentlicher Faktor für eine bessere Effizienz aller Algorithmen. Zusätzlich hat sich \textsc{Filter-Kruskal} als eine sehr effiziente Option offenbart um in einem \boruvkaStep parallele Kanten zu entfernen.\\
Zusätzlich konnten wir beobachten, dass die Überlagerung von Nachrichten ein effizientes Vorgehen sein kann, um einen \boruvkaStep weiter zu beschleunigen.
Allerdings konnten wir aus Zeitgründen dieses Vorgehen nur Anhand einer beispielhaften Implementierung betrachten, die in der Praxis nicht sinnvoll ist.
Außerdem ist eine Nachrichtenüberlagerung nur sinnvoll wenn die lokalen MST Berechnungen einen ähnlich Anteil an der Laufzeit ausmachen wie der Allreduce Aufruf. Dies war bei unseren Experimenten nur auf weniger dichten Graphen möglich, was wiederum die generelle Effizienz der Algorithmen mindert.


Da dieser Ansatz dennoch vielversprechend erscheint, ist eine weiterführende Forschung in diesem Bereich notwendig.
Beispielsweise könnte eine Implementierung mittels Hyper-Threading zeigen, wie effizient dieser Ansatz in der Praxis wirklich ist.
Zusätzlich sollten weitere Experimente folgen, die zeigen welche Eingaben von der Nachrichtenüberlagerung am meisten profitieren.







%%%%%%%%%%%%%%%%%%%%%%%%%%%%%%%%%%%%%%%%%%%%%%%%%%%%%%%%%%%%%%%%%%%%%%
\listoffigures
%\listoftables
\listofalgorithms

\clearpage

%%%%%%%%%%%%%%%%%%%%%%%%%%%%%%%%%%%%%%%%%%%%%%%%%%%%%%%%%%%%%%%%%%%%%%



\printbibliography


\end{document}
